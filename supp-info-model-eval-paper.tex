% Options for packages loaded elsewhere
% Options for packages loaded elsewhere
\PassOptionsToPackage{unicode}{hyperref}
\PassOptionsToPackage{hyphens}{url}
%
\documentclass[
  letterpaper,
  DIV=11,
  numbers=noendperiod]{scrartcl}
\usepackage{xcolor}
\usepackage{amsmath,amssymb}
\setcounter{secnumdepth}{5}
\usepackage{iftex}
\ifPDFTeX
  \usepackage[T1]{fontenc}
  \usepackage[utf8]{inputenc}
  \usepackage{textcomp} % provide euro and other symbols
\else % if luatex or xetex
  \usepackage{unicode-math} % this also loads fontspec
  \defaultfontfeatures{Scale=MatchLowercase}
  \defaultfontfeatures[\rmfamily]{Ligatures=TeX,Scale=1}
\fi
\usepackage{lmodern}
\ifPDFTeX\else
  % xetex/luatex font selection
\fi
% Use upquote if available, for straight quotes in verbatim environments
\IfFileExists{upquote.sty}{\usepackage{upquote}}{}
\IfFileExists{microtype.sty}{% use microtype if available
  \usepackage[]{microtype}
  \UseMicrotypeSet[protrusion]{basicmath} % disable protrusion for tt fonts
}{}
\usepackage{setspace}
\makeatletter
\@ifundefined{KOMAClassName}{% if non-KOMA class
  \IfFileExists{parskip.sty}{%
    \usepackage{parskip}
  }{% else
    \setlength{\parindent}{0pt}
    \setlength{\parskip}{6pt plus 2pt minus 1pt}}
}{% if KOMA class
  \KOMAoptions{parskip=half}}
\makeatother
% Make \paragraph and \subparagraph free-standing
\makeatletter
\ifx\paragraph\undefined\else
  \let\oldparagraph\paragraph
  \renewcommand{\paragraph}{
    \@ifstar
      \xxxParagraphStar
      \xxxParagraphNoStar
  }
  \newcommand{\xxxParagraphStar}[1]{\oldparagraph*{#1}\mbox{}}
  \newcommand{\xxxParagraphNoStar}[1]{\oldparagraph{#1}\mbox{}}
\fi
\ifx\subparagraph\undefined\else
  \let\oldsubparagraph\subparagraph
  \renewcommand{\subparagraph}{
    \@ifstar
      \xxxSubParagraphStar
      \xxxSubParagraphNoStar
  }
  \newcommand{\xxxSubParagraphStar}[1]{\oldsubparagraph*{#1}\mbox{}}
  \newcommand{\xxxSubParagraphNoStar}[1]{\oldsubparagraph{#1}\mbox{}}
\fi
\makeatother


\usepackage{longtable,booktabs,array}
\usepackage{calc} % for calculating minipage widths
% Correct order of tables after \paragraph or \subparagraph
\usepackage{etoolbox}
\makeatletter
\patchcmd\longtable{\par}{\if@noskipsec\mbox{}\fi\par}{}{}
\makeatother
% Allow footnotes in longtable head/foot
\IfFileExists{footnotehyper.sty}{\usepackage{footnotehyper}}{\usepackage{footnote}}
\makesavenoteenv{longtable}
\usepackage{graphicx}
\makeatletter
\newsavebox\pandoc@box
\newcommand*\pandocbounded[1]{% scales image to fit in text height/width
  \sbox\pandoc@box{#1}%
  \Gscale@div\@tempa{\textheight}{\dimexpr\ht\pandoc@box+\dp\pandoc@box\relax}%
  \Gscale@div\@tempb{\linewidth}{\wd\pandoc@box}%
  \ifdim\@tempb\p@<\@tempa\p@\let\@tempa\@tempb\fi% select the smaller of both
  \ifdim\@tempa\p@<\p@\scalebox{\@tempa}{\usebox\pandoc@box}%
  \else\usebox{\pandoc@box}%
  \fi%
}
% Set default figure placement to htbp
\def\fps@figure{htbp}
\makeatother


% definitions for citeproc citations
\NewDocumentCommand\citeproctext{}{}
\NewDocumentCommand\citeproc{mm}{%
  \begingroup\def\citeproctext{#2}\cite{#1}\endgroup}
\makeatletter
 % allow citations to break across lines
 \let\@cite@ofmt\@firstofone
 % avoid brackets around text for \cite:
 \def\@biblabel#1{}
 \def\@cite#1#2{{#1\if@tempswa , #2\fi}}
\makeatother
\newlength{\cslhangindent}
\setlength{\cslhangindent}{1.5em}
\newlength{\csllabelwidth}
\setlength{\csllabelwidth}{3em}
\newenvironment{CSLReferences}[2] % #1 hanging-indent, #2 entry-spacing
 {\begin{list}{}{%
  \setlength{\itemindent}{0pt}
  \setlength{\leftmargin}{0pt}
  \setlength{\parsep}{0pt}
  % turn on hanging indent if param 1 is 1
  \ifodd #1
   \setlength{\leftmargin}{\cslhangindent}
   \setlength{\itemindent}{-1\cslhangindent}
  \fi
  % set entry spacing
  \setlength{\itemsep}{#2\baselineskip}}}
 {\end{list}}
\usepackage{calc}
\newcommand{\CSLBlock}[1]{\hfill\break\parbox[t]{\linewidth}{\strut\ignorespaces#1\strut}}
\newcommand{\CSLLeftMargin}[1]{\parbox[t]{\csllabelwidth}{\strut#1\strut}}
\newcommand{\CSLRightInline}[1]{\parbox[t]{\linewidth - \csllabelwidth}{\strut#1\strut}}
\newcommand{\CSLIndent}[1]{\hspace{\cslhangindent}#1}



\setlength{\emergencystretch}{3em} % prevent overfull lines

\providecommand{\tightlist}{%
  \setlength{\itemsep}{0pt}\setlength{\parskip}{0pt}}



 


\KOMAoption{captions}{tableheading}
\renewcommand{\thefigure}{S.\arabic{figure}}
\setcounter{figure}{0}
\renewcommand{\thetable}{S.\arabic{table}}
\setcounter{table}{0}
\makeatletter
\@ifpackageloaded{caption}{}{\usepackage{caption}}
\AtBeginDocument{%
\ifdefined\contentsname
  \renewcommand*\contentsname{Table of contents}
\else
  \newcommand\contentsname{Table of contents}
\fi
\ifdefined\listfigurename
  \renewcommand*\listfigurename{List of Figures}
\else
  \newcommand\listfigurename{List of Figures}
\fi
\ifdefined\listtablename
  \renewcommand*\listtablename{List of Tables}
\else
  \newcommand\listtablename{List of Tables}
\fi
\ifdefined\figurename
  \renewcommand*\figurename{Figure}
\else
  \newcommand\figurename{Figure}
\fi
\ifdefined\tablename
  \renewcommand*\tablename{Table}
\else
  \newcommand\tablename{Table}
\fi
}
\@ifpackageloaded{float}{}{\usepackage{float}}
\floatstyle{ruled}
\@ifundefined{c@chapter}{\newfloat{codelisting}{h}{lop}}{\newfloat{codelisting}{h}{lop}[chapter]}
\floatname{codelisting}{Listing}
\newcommand*\listoflistings{\listof{codelisting}{List of Listings}}
\makeatother
\makeatletter
\makeatother
\makeatletter
\@ifpackageloaded{caption}{}{\usepackage{caption}}
\@ifpackageloaded{subcaption}{}{\usepackage{subcaption}}
\makeatother
\usepackage{bookmark}
\IfFileExists{xurl.sty}{\usepackage{xurl}}{} % add URL line breaks if available
\urlstyle{same}
\hypersetup{
  pdftitle={Supporting Information for Evaluation of New Snow Interception and Canopy Snow Ablation Parameterisations for Partitioning Snowfall in Needleleaf Forests},
  hidelinks,
  pdfcreator={LaTeX via pandoc}}


\title{Supporting Information for Evaluation of New Snow Interception
and Canopy Snow Ablation Parameterisations for Partitioning Snowfall in
Needleleaf Forests}
\author{}
\date{}
\begin{document}
\maketitle


\setstretch{1.5}
\textbf{Authors:}

Alex C. Cebulski\textsuperscript{1} (ORCID ID - 0000-0001-7910-5056)

John W. Pomeroy\textsuperscript{1} (ORCID ID - 0000-0002-4782-7457)

Bill C. Floyd\textsuperscript{2,3}

\textsuperscript{1}Centre for Hydrology, University of Saskatchewan,
Canmore, Canada\\
\textsuperscript{2}Ministry of Forests, Government of British Columbia,
Nanaimo, British Columbia\\
\textsuperscript{3}Coastal Hydrology Research Lab, Vancouver Island
University, Nanaimo, British Columbia

\textbf{Corresponding Author:} A.C. Cebulski, alex.cebulski@usask.ca

\pagebreak

\section{Description of New Modules added to the Cold Regions
Hydrological Modelling Platform
(CRHM)}\label{description-of-new-modules-added-to-the-cold-regions-hydrological-modelling-platform-crhm}

The following describes the new modules added to the CRHM software as
part of this research. The new modules are fully implemented in the CRHM
repository
\href{https://github.com/acebulsk/crhmcode/blob/alex-ceb-thesis-changes}{\color{blue}{here}}
and specific changes are documented in pull request
\href{https://github.com/srlabUsask/crhmcode/pull/471}{\color{blue}{\#471}}.

\subsection{Initial Snow Interception}\label{initial-snow-interception}

\textbf{Description:}

A new module called \texttt{CanopyVectorBased} was added to CRHM to
compute initial snow interception in the canopy as described in Cebulski
\& Pomeroy (2025a). This module differs from the existing CRHM
\texttt{CanopyClearingGap} module based on Ellis et al. (2010) which
calculated initial interception as a function of antecedent canopy snow
load and canopy density. The new module represents an improvement by
calculating initial interception as a function of snowfall and canopy
density (function of canopy coverage and hydrometeor trajectory angle)
based on observations presented in Cebulski \& Pomeroy (2025a). The
association of interception with canopy snow load is now handled only
once by the canopy snow unloading routine in the canopy snow ablation
module.

\textbf{Module Code:}

\begin{itemize}
\tightlist
\item
  https://github.com/acebulsk/crhmcode/blob/alex-ceb-thesis-changes/crhmcode/src/modules/ClassCRHMCanopyVectorBased.cpp
\item
  https://github.com/acebulsk/crhmcode/blob/alex-ceb-thesis-changes/crhmcode/src/modules/ClassCRHMCanopyVectorBased.h
\end{itemize}

\subsection{Canopy Snow Energy and Mass
Balance}\label{canopy-snow-energy-and-mass-balance}

\textbf{Description:}

A new module called \texttt{CanopySnowBalance} simulates the
physically-based energy and mass balance of snow stored in forest
canopies following developments from Cebulski \& Pomeroy (2025b). It is
designed to be paired with the \texttt{CanopyVectorBased} module and
together with \texttt{CanopySnowBalance} replaces the
\texttt{CanopyClearingGap} module in CRHM. The
\texttt{CanopySnowBalance} represents both dry- and melt-induced canopy
snow unloading coupled with energy balance-based melt and sublimation
calculations.

\textbf{Module Code:}

\begin{itemize}
\tightlist
\item
  https://github.com/acebulsk/crhmcode/blob/alex-ceb-thesis-changes/crhmcode/src/modules/ClassCanopySnowBalanceCRHM.cpp
\item
  https://github.com/acebulsk/crhmcode/blob/alex-ceb-thesis-changes/crhmcode/src/modules/ClassCanopySnowBalanceCRHM.h
\item
  https://github.com/acebulsk/crhmcode/blob/alex-ceb-thesis-changes/crhmcode/src/modules/ClassCanopySnowBalanceBase.cpp
\item
  https://github.com/acebulsk/crhmcode/blob/alex-ceb-thesis-changes/crhmcode/src/modules/ClassCanopySnowBalanceBase.h
\end{itemize}

\section{Canopy Snow Load Evaluation}\label{canopy-snow-load-evaluation}

Figure~\ref{fig-swe-ann-peak} shows a subset of Figure 5 from the main
manuscript to better illustrate select spring snow events with mixed
snow/rain.

\begin{figure}[H]

\centering{

\pandocbounded{\includegraphics[keepaspectratio]{figs/supplement/cpy_load_2019_spring_events.png}}

}

\caption{\label{fig-swe-ann-peak}Timeseries of simulated (CP25 and E10)
and observed (weighed tree) canopy load at Marmot Creek for select
spring snow/rain events between May and June 2019.}

\end{figure}%

\section{Canopy Snow Load Simulation}\label{canopy-snow-load-simulation}

Figure~\ref{fig-cpy-swe-select} shows a subset of Figure 11 from the
main manuscript to illustrate the difference in initial accumulation of
snowfall in the canopy between the two models.

\begin{figure}[H]

\centering{

\pandocbounded{\includegraphics[keepaspectratio]{figs/final/crhm_cpy_swe_select_yrs_baseline_vs_cansnobal.png}}

}

\caption{\label{fig-cpy-swe-select}Timeseries of simulated canopy load
for CP25 and E10 at each station for select water years. The water year
2017 was selected for Fortress, Marmot, and Wolf Creek, while 2007 was
selected for Russell.}

\end{figure}%

\pagebreak

\section*{References}\label{references}
\addcontentsline{toc}{section}{References}

\phantomsection\label{refs}
\begin{CSLReferences}{1}{0}
\bibitem[\citeproctext]{ref-Cebulski2025a}
Cebulski, A. C., \& Pomeroy, J. W. (2025a). Snow interception
relationships with meteorology and canopy density. \emph{Hydrological
Processes}, \emph{39}(4), e70135.
\url{https://doi.org/10.1002/hyp.70135}

\bibitem[\citeproctext]{ref-Cebulski2025b}
Cebulski, A. C., \& Pomeroy, J. W. (2025b). Processes governing the
ablation of intercepted snow. \emph{Water Resources Research}, \emph{in
review}.

\bibitem[\citeproctext]{ref-Ellis2010}
Ellis, C. R., Pomeroy, J. W., Brown, T., \& MacDonald, J. (2010).
Simulation of snow accumulation and melt in needleleaf forest
environments. \emph{Hydrology and Earth System Sciences}, \emph{14}(6),
925--940. \url{https://doi.org/10.5194/hess-14-925-2010}

\end{CSLReferences}




\end{document}
