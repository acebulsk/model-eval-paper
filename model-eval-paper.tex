% Options for packages loaded elsewhere
% Options for packages loaded elsewhere
\PassOptionsToPackage{unicode}{hyperref}
\PassOptionsToPackage{hyphens}{url}
%
\documentclass[
  letterpaper,
  DIV=11,
  numbers=noendperiod]{scrartcl}
\usepackage{xcolor}
\usepackage{amsmath,amssymb}
\setcounter{secnumdepth}{5}
\usepackage{iftex}
\ifPDFTeX
  \usepackage[T1]{fontenc}
  \usepackage[utf8]{inputenc}
  \usepackage{textcomp} % provide euro and other symbols
\else % if luatex or xetex
  \usepackage{unicode-math} % this also loads fontspec
  \defaultfontfeatures{Scale=MatchLowercase}
  \defaultfontfeatures[\rmfamily]{Ligatures=TeX,Scale=1}
\fi
\usepackage{lmodern}
\ifPDFTeX\else
  % xetex/luatex font selection
\fi
% Use upquote if available, for straight quotes in verbatim environments
\IfFileExists{upquote.sty}{\usepackage{upquote}}{}
\IfFileExists{microtype.sty}{% use microtype if available
  \usepackage[]{microtype}
  \UseMicrotypeSet[protrusion]{basicmath} % disable protrusion for tt fonts
}{}
\usepackage{setspace}
\makeatletter
\@ifundefined{KOMAClassName}{% if non-KOMA class
  \IfFileExists{parskip.sty}{%
    \usepackage{parskip}
  }{% else
    \setlength{\parindent}{0pt}
    \setlength{\parskip}{6pt plus 2pt minus 1pt}}
}{% if KOMA class
  \KOMAoptions{parskip=half}}
\makeatother
% Make \paragraph and \subparagraph free-standing
\makeatletter
\ifx\paragraph\undefined\else
  \let\oldparagraph\paragraph
  \renewcommand{\paragraph}{
    \@ifstar
      \xxxParagraphStar
      \xxxParagraphNoStar
  }
  \newcommand{\xxxParagraphStar}[1]{\oldparagraph*{#1}\mbox{}}
  \newcommand{\xxxParagraphNoStar}[1]{\oldparagraph{#1}\mbox{}}
\fi
\ifx\subparagraph\undefined\else
  \let\oldsubparagraph\subparagraph
  \renewcommand{\subparagraph}{
    \@ifstar
      \xxxSubParagraphStar
      \xxxSubParagraphNoStar
  }
  \newcommand{\xxxSubParagraphStar}[1]{\oldsubparagraph*{#1}\mbox{}}
  \newcommand{\xxxSubParagraphNoStar}[1]{\oldsubparagraph{#1}\mbox{}}
\fi
\makeatother


\usepackage{longtable,booktabs,array}
\usepackage{calc} % for calculating minipage widths
% Correct order of tables after \paragraph or \subparagraph
\usepackage{etoolbox}
\makeatletter
\patchcmd\longtable{\par}{\if@noskipsec\mbox{}\fi\par}{}{}
\makeatother
% Allow footnotes in longtable head/foot
\IfFileExists{footnotehyper.sty}{\usepackage{footnotehyper}}{\usepackage{footnote}}
\makesavenoteenv{longtable}
\usepackage{graphicx}
\makeatletter
\newsavebox\pandoc@box
\newcommand*\pandocbounded[1]{% scales image to fit in text height/width
  \sbox\pandoc@box{#1}%
  \Gscale@div\@tempa{\textheight}{\dimexpr\ht\pandoc@box+\dp\pandoc@box\relax}%
  \Gscale@div\@tempb{\linewidth}{\wd\pandoc@box}%
  \ifdim\@tempb\p@<\@tempa\p@\let\@tempa\@tempb\fi% select the smaller of both
  \ifdim\@tempa\p@<\p@\scalebox{\@tempa}{\usebox\pandoc@box}%
  \else\usebox{\pandoc@box}%
  \fi%
}
% Set default figure placement to htbp
\def\fps@figure{htbp}
\makeatother


% definitions for citeproc citations
\NewDocumentCommand\citeproctext{}{}
\NewDocumentCommand\citeproc{mm}{%
  \begingroup\def\citeproctext{#2}\cite{#1}\endgroup}
\makeatletter
 % allow citations to break across lines
 \let\@cite@ofmt\@firstofone
 % avoid brackets around text for \cite:
 \def\@biblabel#1{}
 \def\@cite#1#2{{#1\if@tempswa , #2\fi}}
\makeatother
\newlength{\cslhangindent}
\setlength{\cslhangindent}{1.5em}
\newlength{\csllabelwidth}
\setlength{\csllabelwidth}{3em}
\newenvironment{CSLReferences}[2] % #1 hanging-indent, #2 entry-spacing
 {\begin{list}{}{%
  \setlength{\itemindent}{0pt}
  \setlength{\leftmargin}{0pt}
  \setlength{\parsep}{0pt}
  % turn on hanging indent if param 1 is 1
  \ifodd #1
   \setlength{\leftmargin}{\cslhangindent}
   \setlength{\itemindent}{-1\cslhangindent}
  \fi
  % set entry spacing
  \setlength{\itemsep}{#2\baselineskip}}}
 {\end{list}}
\usepackage{calc}
\newcommand{\CSLBlock}[1]{\hfill\break\parbox[t]{\linewidth}{\strut\ignorespaces#1\strut}}
\newcommand{\CSLLeftMargin}[1]{\parbox[t]{\csllabelwidth}{\strut#1\strut}}
\newcommand{\CSLRightInline}[1]{\parbox[t]{\linewidth - \csllabelwidth}{\strut#1\strut}}
\newcommand{\CSLIndent}[1]{\hspace{\cslhangindent}#1}



\setlength{\emergencystretch}{3em} % prevent overfull lines

\providecommand{\tightlist}{%
  \setlength{\itemsep}{0pt}\setlength{\parskip}{0pt}}



 


\KOMAoption{captions}{tableheading}
\usepackage[left]{lineno}
\linenumbers
\usepackage{xcolor}
\usepackage{float}
\floatplacement{table}{H}
\usepackage{setspace}
\makeatletter
\@ifpackageloaded{caption}{}{\usepackage{caption}}
\AtBeginDocument{%
\ifdefined\contentsname
  \renewcommand*\contentsname{Table of contents}
\else
  \newcommand\contentsname{Table of contents}
\fi
\ifdefined\listfigurename
  \renewcommand*\listfigurename{List of Figures}
\else
  \newcommand\listfigurename{List of Figures}
\fi
\ifdefined\listtablename
  \renewcommand*\listtablename{List of Tables}
\else
  \newcommand\listtablename{List of Tables}
\fi
\ifdefined\figurename
  \renewcommand*\figurename{Figure}
\else
  \newcommand\figurename{Figure}
\fi
\ifdefined\tablename
  \renewcommand*\tablename{Table}
\else
  \newcommand\tablename{Table}
\fi
}
\@ifpackageloaded{float}{}{\usepackage{float}}
\floatstyle{ruled}
\@ifundefined{c@chapter}{\newfloat{codelisting}{h}{lop}}{\newfloat{codelisting}{h}{lop}[chapter]}
\floatname{codelisting}{Listing}
\newcommand*\listoflistings{\listof{codelisting}{List of Listings}}
\makeatother
\makeatletter
\usepackage{pdflscape}
\makeatother
\makeatletter
\makeatother
\makeatletter
\@ifpackageloaded{caption}{}{\usepackage{caption}}
\@ifpackageloaded{subcaption}{}{\usepackage{subcaption}}
\makeatother
\usepackage{bookmark}
\IfFileExists{xurl.sty}{\usepackage{xurl}}{} % add URL line breaks if available
\urlstyle{same}
\hypersetup{
  pdftitle={Evaluation of New Snow Interception and Canopy Snow Ablation Parameterisations for Partitioning Snowfall in Needleleaf Forests},
  pdfauthor={Alex C. Cebulski1, John W. Pomeroy1 Bill C. Floyd2,3},
  hidelinks,
  pdfcreator={LaTeX via pandoc}}


\title{Evaluation of New Snow Interception and Canopy Snow Ablation
Parameterisations for Partitioning Snowfall in Needleleaf Forests}
\author{Alex C. Cebulski\textsuperscript{1}, John W.
Pomeroy\textsuperscript{1} Bill C. Floyd\textsuperscript{2,3}}
\date{}
\begin{document}
\maketitle


\setstretch{1.5}
\textsuperscript{1}Centre for Hydrology, University of Saskatchewan,
Canmore, Canada\\
\textsuperscript{2}Ministry of Forests, Provincial Government of British
Columbia, Nanaimo, British Columbia\\
\textsuperscript{3}Coastal Hydrology Research Lab, Vancouver Island
University, Nanaimo, British Columbia

\textbf{Corresponding Author:} Alex C. Cebulski (alex.cebulski@usask.ca,
ORCID ID - 0000-0001-7910-5056)

\textbf{Abstract}

Snow falls in forests over 23\% of the global land mass where snow
interception and canopy snow ablation processes influence snow
accumulation and land-surface energy exchanges. These processes are
strongly influenced by both meteorological conditions and canopy
density, resulting in differing process emergence in existing theories
developed in distinctive climates, seasons, and forest types, and
limited transferability to new environments. Recent studies have
revealed new relationships to represent snow interception and canopy
snow ablation processes applicable across a broader range of canopy
structures and climatic conditions. To assess the effectiveness of these
new parameterisations across differing climate and forests, both novel
and traditional routines were implemented in the Cold Regions
Hydrological Modelling platform and evaluated against observations of
canopy and subcanopy snow water equivalent across four Canadian sites:
two continental climate sites (Marmot Creek and Fortress Mountain,
Alberta), one subarctic site (Wolf Creek, Yukon Territory), and one
temperature-maritime site (Russell Creek, British Columbia). The
observed fraction of seasonal snowfall stored in the subcanopy snowpack
at peak SWE varied from 0.3 at Russell Creek, 0.4 at Marmot and Wolf
Creek, and 0.6 at Fortress Mountain. Uncalibrated simulation of canopy
intercepted snow duration at Marmot Creek improved with the new model,
where the fractional differences from observations decreased from
-40.1\% to 10.2\%. The new model demonstrated substantially improved
simulation of subcanopy snow accumulation, with mean bias dropping from
-24.5 to -1.68 kg m\textsuperscript{-2} across the four sites. At cold,
low-wind sites, about half of annual snowfall was lost via sublimation
of intercepted snow, whereas greater unloading at a colder, wind-exposed
site reduced sublimation losses. At the high-snowfall temperate maritime
site, canopy snowmelt, meltwater drip, and melt-induced unloading
dominated, delivering the largest fraction of snowfall to the forest
floor despite high initial interception efficiency.

\pagebreak

\section{Introduction}\label{introduction}

Snow is an important water resource, directly supporting over two
billion people globally (Immerzeel et al., 2020; Viviroli et al., 2020),
while also affecting the Earth's energy balance via surface albedo
(Thackeray et al., 2014; Wang et al., 2016), surface temperature
(Pomeroy et al., 2016), soil temperature (Zhang et al., 2018), and
stream temperatures (Leach \& Moore, 2014). However, snowpacks are
increasingly threatened due to changes in both climate and vegetation
cover worldwide (Immerzeel et al., 2020; López-Moreno et al., 2014;
Viviroli et al., 2020). Hydrological models are essential tools for
understanding how climate and vegetation influence snow processes and
downstream water resources, and their accuracy depends on accurate
representations of forest-snow processes. Snow falls in forested areas
over half of the Northern Hemisphere (Kim et al., 2017) and over 23\% of
land mass globally (Deschamps-Berger et al., 2025), spanning diverse
climates and forest structures, highlighting the need for robust,
transferable models. In cold-dry climates, sublimation of snow
intercepted by forest canopies can return up to 45\% of seasonal
snowfall back to the atmosphere (Essery et al., 2003;
Sanmiguel-Vallelado et al., 2017), whereas in temperate-maritime
climates, sublimation is less prevalent and a large fraction of snowfall
melts in the canopy (Storck et al., 2002). Yet, uncertainties in
forest-snow process representation lead to variable transferability
across climates and forest types when simulating subcanopy snow water
equivalent (SWE) (Essery et al., 2003; Gelfan et al., 2004; Krinner et
al., 2018; Rutter et al., 2009) and diagnosing snow processes (Lumbrazo
et al., 2022; Lundquist et al., 2021). The strong dependence of snowfall
partitioning on meteorology and canopy density compounds this
variability (Mazzotti et al., 2021; Pomeroy et al., 1998; Rojas-Heredia
et al., 2024; Staines \& Pomeroy, 2023; Storck et al., 2002),
challenging earlier canopy snow parameterisations developed from limited
observations and leading to distinct process emergence across differing
environmental conditions (Lundquist et al., 2021). While simulating SWE
in forests remains challenging, it is a crucial aspect to understanding
the impacts of climate and land cover changes on water resources in many
forested cold regions across the globe.

Recent studies have advanced understanding of the canopy snow energy and
mass balance across a broader range in meteorological conditions and
canopy structures (Cebulski \& Pomeroy, 2025b, 2025c; Lumbrazo et al.,
2022; Lundquist et al., 2021; Mazzotti et al., 2021), with potential to
improve SWE simulations in more diverse forested basins. For example,
Lundquist et al. (2021) demonstrated that calculating throughfall as a
function of antecedent snow load can overestimate the amount of snow
reaching the ground---when also combined with a comprehensive canopy
snow unloading routine. Building on this, Staines \& Pomeroy (2023) and
Cebulski \& Pomeroy (2025b) showed that initial interception can be
predicted as a function of canopy density without assuming a maximum
canopy snow load. Moreover, Roesch et al. (2001) and Lumbrazo et al.
(2022) showed the importance of representing both wind and melt-induced
unloading for representing canopy snow ablation. A new physically-based
canopy snow mass and energy balance developed by Cebulski \& Pomeroy
(2025c) provided improved representation of canopy snow ablation
compared to previous approaches that were either missing key processes
such as dry snow unloading (Andreadis et al., 2009) or were based on
empirical relationships such as ice-bulb temperature indexed melt
unloading and drip (Ellis et al., 2010). These advances have been
implemented as new parameterisations in the modular Cold Regions
Hydrological Modelling Platform (Pomeroy et al., 2022) to answer the
following research questions:

\begin{enumerate}
\def\labelenumi{\arabic{enumi}.}
\tightlist
\item
  What is the fraction of seasonal snowfall stored in the subcanopy
  snowpack across forests with differing climate and forest types?
\item
  How accurately does a novel hydrological model simulate canopy snow
  load and subcanopy SWE across varying forest types and climates
  compared to an existing model?
\item
  How is snowfall in needleleaf forests partitioned into interception,
  sublimation, unloading, and melt/drip in differing climates and canopy
  structures?
\end{enumerate}

The objective of this research is to evaluate new snow interception and
ablation parameterisations for simulating canopy and subcanopy SWE and
to use the new parameterisations to diagnose the role of these processes
in partitioning snowfall and governing snow accumulation under
needleleaf forest canopies. Evaluation of the new model in simulating
initial snow interception has been addressed in Cebulski \& Pomeroy
(2025b) and canopy snow ablation in Cebulski \& Pomeroy (2025c).

\section{Methods}\label{methods}

\subsection{Study Sites}\label{study-sites}

The model evaluation was conducted at four locations in western and
northern Canada spanning a range of climate and forest types
(Fig.~\ref{fig-map}; Table~\ref{tbl-site-meta}). The model simulation
years for each site are shown in Table~\ref{tbl-site-meta} and were
selected based on the availability of subcanopy SWE measurements and
hourly station-based observations of air temperature, relative humidity,
wind speed, total precipitation, and net solar radiation adjacent to the
snow survey transects. At each site, snow surveys consisted of snow
depth measurements at all locations and snow density measurements at one
out of every five locations. SWE was calculated from snow depth and snow
density following the methods outlined in Pomeroy \& Gray (1995). The
four study sites include:

Wolf Creek Research Basin - Forest Site (60.60°N, 134.96°W, 750 m asl.)
is located 16 km south of Whitehorse, Yukon Territory in a level dense
forest with a sub-arctic climate (see basin scale location in Fig. 1 in
Rasouli et al., 2019). Snow surveys were conducted along a transect that
traverses through mature forest consisting of primarily white spruce
(\emph{Picea glauca}) and interior Lodgepole pine (\emph{Pinus
contorta}). Additional details on the snow survey and meteorological
measurements is described in Rasouli et al. (2019).

Russell Creek Experimental Watershed - Upper Stephanie Old Growth Site
(50.32°N, 126.35°W, 700 m asl.) is located on northern Vancouver Island,
British Columbia (Fig.~\ref{fig-map}) in a temperate-maritime climate
that receives substantial precipitation (\textgreater2000 mm/yr,
Fig.~\ref{fig-met-normals}). Snow survey transects were conducted in
cardinal directions within a mature old growth forest that consists of
Amabilis fir (\emph{Abies amabilis}) and western hemlock (\emph{Tsuga
heterophylla}) (Floyd, 2012). Additional details on the snow survey and
meteorological instrumentation are provided in Floyd (2012). Total
precipitation data were unavailable at the Russell site for the 2008
water year. For this period, records from the Tsitika Summit station
(50.28°N, 126.36°W; 450 m asl.), operated by the British Columbia
Ministry of Transport and located 5 km from Russell, were used instead.

Fortress Mountain Research Basin - Powerline site (50.83°N, 115.20°W,
2100 m asl., Kananaskis, Alberta) is located on a wind-exposed subalpine
ridge top covered with sparse forest with a continental climate
(Fig.~\ref{fig-map}). The vegetation at this site consists of coexisting
mature subalpine fir (\emph{Abies lasiocarpa}) and Engelmann spruce
(\emph{Picea engelmannii}) tree species (Langs et al., 2020). Snow
survey measurements of snow depth and density were collected following a
transect through mature forest east of the Powerline meteorological
tower (see Fig. 1 in Cebulski \& Pomeroy, 2025b). The meteorological
forcing data used in this study is described in detail in Cebulski \&
Pomeroy (2025c). An evaluation of the new canopy snow model on canopy
snow load observations was presented in Cebulski \& Pomeroy (2025b) and
Cebulski \& Pomeroy (2025c) at Fortress Mountain.

Marmot Creek Research Basin - Upper Forest site (50.93°N, 115.16°W, 1848
m asl., Kananaskis, Alberta) is located on a dense forested plateau with
a continental climate 14 km north of Fortress (see basin scale location
in Fig. 1 in Fang et al., 2019) but receives much less precipitation
(Fig.~\ref{fig-met-normals}). The forest consists primarily of Engelmann
spruce (\emph{Picea engelmannii}), subalpine fir (\emph{Abies
lasiocarpa}), and Lodgepole pine (\emph{Pinus contorta}) species (Fang
et al., 2019; Staines \& Pomeroy, 2023). Snow surveys were conducted
following a cardinal transect through mature forest surrounding the
``Upper Clearing'' meteorological tower (see Fig. 1b in Staines \&
Pomeroy, 2023). The meteorological forcing data and corresponding
instrumentation used from this site are described in Fang et al. (2019).
Canopy snow load observations were collected within the Upper Forest to
validate simulations. A weighed subcanopy snow bucket installed by
MacDonald (2010) provided throughfall measurements, which were used with
a mass balance calculation (Eq. 5 in Cebulski \& Pomeroy, 2025a) to
estimate canopy snow load for 11 snowfall events between February 2007
and February 2008. A weighed tree lysimeter installed by Staines (2021)
on the Upper Forest meteorological tower provided measurements of snow
load between December 2018 and June 2019. The weighed tree was scaled
from weight of intercepted snow in the tree to mass of intercepted snow
per unit area using snow survey measurements from the two snowfall
events reported in Staines \& Pomeroy (2023) following the methodology
outlined in Pomeroy \& Schmidt (1993) and Hedstrom \& Pomeroy (1998).

\begin{landscape}

\begin{longtable}[]{@{}
  >{\raggedright\arraybackslash}p{(\linewidth - 16\tabcolsep) * \real{0.1200}}
  >{\raggedright\arraybackslash}p{(\linewidth - 16\tabcolsep) * \real{0.0600}}
  >{\raggedleft\arraybackslash}p{(\linewidth - 16\tabcolsep) * \real{0.1400}}
  >{\raggedleft\arraybackslash}p{(\linewidth - 16\tabcolsep) * \real{0.1400}}
  >{\raggedleft\arraybackslash}p{(\linewidth - 16\tabcolsep) * \real{0.1600}}
  >{\raggedleft\arraybackslash}p{(\linewidth - 16\tabcolsep) * \real{0.0600}}
  >{\raggedleft\arraybackslash}p{(\linewidth - 16\tabcolsep) * \real{0.0700}}
  >{\raggedleft\arraybackslash}p{(\linewidth - 16\tabcolsep) * \real{0.0800}}
  >{\raggedright\arraybackslash}p{(\linewidth - 16\tabcolsep) * \real{0.1900}}@{}}

\caption{\label{tbl-site-meta}Simulation period (Years), location, and
vegetation characteristics, including canopy cover (\(C_c\)), leaf area
index (LAI), and mean tree height (\(\overline{h_t}\)), for the four
study sites.}

\tabularnewline

\toprule\noalign{}
\begin{minipage}[b]{\linewidth}\raggedright
Site Name
\end{minipage} & \begin{minipage}[b]{\linewidth}\raggedright
Years
\end{minipage} & \begin{minipage}[b]{\linewidth}\raggedleft
Elevation (m)
\end{minipage} & \begin{minipage}[b]{\linewidth}\raggedleft
Latitude (°N)
\end{minipage} & \begin{minipage}[b]{\linewidth}\raggedleft
Longitude (°W)
\end{minipage} & \begin{minipage}[b]{\linewidth}\raggedleft
\(C_c\) (-)
\end{minipage} & \begin{minipage}[b]{\linewidth}\raggedleft
LAI (-)
\end{minipage} & \begin{minipage}[b]{\linewidth}\raggedleft
\(\overline{h_t}\) (m)
\end{minipage} & \begin{minipage}[b]{\linewidth}\raggedright
Dominant Species
\end{minipage} \\
\midrule\noalign{}
\endhead
\bottomrule\noalign{}
\endlastfoot
Wolf Creek & 2015--2022 & 750 & 60.60 & 134.96 & 0.81 & 3.82 & 15.0 &
White Spruce and interior lodgepole pine \\
Marmot Creek & 2006--2023 & 1848 & 50.93 & 115.16 & 0.80 & 3.00 & 15.0 &
Engelmann spruce, subalpine fir, and lodgepole pine \\
Fortress Mountain & 2013--2023 & 2100 & 50.83 & 115.20 & 0.65 & 1.44 &
10.5 & Subalpine fir and engelmann spruce \\
Russell Creek & 2006--2008 & 700 & 50.32 & 126.35 & 0.86 & 1.93 & 44.9 &
Amabilis fir and western hemlock \\

\end{longtable}

\end{landscape}

\begin{figure}[H]

\centering{

\includegraphics[width=0.8\linewidth,height=\textheight,keepaspectratio]{figs/final/figure1.png}

}

\caption{\label{fig-map}Map showing the regional scale location of the
four research basins and land cover data from the Canada Centre for
Remote Sensing, Canada Centre for Mapping and Earth Observation \&
Natural Resources Canada (2020) North American Land Change Monitoring
30-metre dataset.}

\end{figure}%

\begin{figure}[H]

\centering{

\pandocbounded{\includegraphics[keepaspectratio]{figs/final/figure2.png}}

}

\caption{\label{fig-met-normals}Mean monthly relative humidity, air
temperature, total precipitation, and wind speed at each station over
the simulation period. Points and solid lines indicate monthly means,
while the shaded band represents the 5th--95th percentile range across
the corresponding years for each site (see Table~\ref{tbl-site-meta} for
station metadata). Observations were not available during the snow free
period for Russell Creek (Jun to Sept). Wind speeds are reported for
above-canopy conditions.}

\end{figure}%

\subsection{Simulation of Subcanopy
Snowpack}\label{simulation-of-subcanopy-snowpack}

The Cold Regions Hydrological Modelling Platform (CRHM) was implemented
to simulate SWE stored in the canopy and in the subcanopy snowpack, and
diagnose processes that partition intercepted snow at each of the four
forest plots. The CRHM platform is described in detail by Pomeroy et al.
(2022). The up-to-date source code is available at
https://github.com/srlabUsask/crhmcode, and the version used in this
manuscript is archived at Pomeroy et al. (2025). Hourly climate forcing
data from station-based measurements of air temperature, relative
humidity, wind speed, total precipitation, and above canopy incoming
solar radiation were used to run the CRHM models. Incoming solar
radiation observations were not available for Wolf Creek and were
simulated following theoretical clear-sky radiation by Garnier \& Ohmura
(1970) and atmospheric transmittance by Shook \& Pomeroy (2011) using an
adaptation of the method developed by Annandale et al. (2002).

Precipitation phase was determined following the psychrometric energy
balance approach of Harder \& Pomeroy (2013) which accounts for the
influence of temperature and humidity on precipitation phase. The canopy
snow mass and energy balance was treated using two different approaches.
An updated approach following new relationships presented in Cebulski \&
Pomeroy (2025b) and Cebulski \& Pomeroy (2025c) hereafter called
``CP25'' to represent the canopy snow energy and mass balance (see
Supporting Information for a description of the changes to CRHM to
implement the new CP25 model). This new approach simulates initial
interception of snow in the canopy as a function of canopy density and
hydrometeor trajectory angle (Cebulski \& Pomeroy, 2025b) and subsequent
ablation of snow intercepted in the canopy by melt and dry-snow
unloading (Cebulski \& Pomeroy, 2025c), energy balance-based snowmelt
(Cebulski \& Pomeroy, 2025c), and energy balance-based sublimation
(Essery et al., 2003). The terminal fall velocity of hydrometeors used
in the initial snow interception parameterisation was assumed to be
constant at 0.8 m s\textsuperscript{-1}, based on observations reported
in Cebulski \& Pomeroy (2025b) and Isyumov (1971). The shear stress used
in the canopy snow unloading parameterisation was approximated as the
square of wind speed multiplied by an empirically derived correction
factor which was obtained from observations at Fortress Mountain
presented in Cebulski \& Pomeroy (2025c). A second approach hereafter
called ``E10'' which is based on observations by Hedstrom \& Pomeroy
(1998), Pomeroy et al. (1998), and Floyd (2012); and implemented as
described in Ellis et al. (2010). E10 calculates initial interception of
snow in the canopy as a function of canopy density, antecedent snow
load, and a species dependent storage capacity following Hedstrom \&
Pomeroy (1998). Ablation of snow intercepted in the canopy is determined
by dry snow unloading (function of canopy snow load as in Hedstrom \&
Pomeroy, 1998), unloading due to melt (Ellis et al., 2010), canopy
snowmelt drainage (threshold function of ice-bulb temperature as in
Ellis et al., 2010), and sublimation by an analytical energy
balance-based parameterisation (Pomeroy et al., 1998). See Cebulski \&
Pomeroy (2025a) for a complete description of the E10 parameterisation.
While neither the E10 nor CP25 model was calibrated for this study,
their parameterisations were originally developed using data from Marmot
and sites in Prince Albert National Park, northern Saskatchewan (E10)
and Fortress (CP25).

A two-layer energy and mass balance snowmelt model (Snobal, Marks et
al., 1998) was used to calculate subcanopy snowpack evolution. Net
shortwave radiation to the subcanopy snowpack was simulated by
calculating the transmittance of irradiance through the canopy, less the
amount reflected from the snow surface (Ellis et al., 2010; Pomeroy et
al., 2008). Incoming longwave radiation to subcanopy snow was simulated
by thermal emissions from the atmosphere and vegetation elements,
weighted by sky-view-factor (Ellis et al., 2010; Pomeroy et al., 2009).
Sensible and latent heat fluxes to the subcanopy snowpack were
determined using an approach adopted from Brutsaert (1982) and Marks \&
Dozier (1992) and is described in the CRHM source code. Only two water
years were simulated at Russell due to limited model forcing and
validation data.

\subsection{Model Evaluation}\label{model-evaluation}

Simulated canopy snow load at Marmot Creek and subcanopy SWE at the four
forest plots by the two models (E10 and CP25) was evaluated using
observations. The performance of the two models was evaluated based on
the differences in simulated (\(S_i\)) and observed (\(O_i\)) values of
SWE using mean bias (MB), root mean squared error (RMSE), Nash-Sutcliffe
efficiency (NSE, Nash \& Sutcliffe, 1970), and Kling-Gupta Efficiency
(KGE, Gupta et al., 2009; Clark et al., 2021) as:

\begin{equation}\phantomsection\label{eq-mb}{
\text{MB} = \frac{1}{n} \sum^n_{i=1}(S_i-O_i)
}\end{equation}

\begin{equation}\phantomsection\label{eq-rmse}{
\text{RMSE} = \sqrt{\frac{\sum^n_{i=1}(S_i-O_i)^2}{n}}
}\end{equation}

\begin{equation}\phantomsection\label{eq-nse}{
\text{NSE} = 1 - \frac{\sum_{i=1}^{n} (S_i - O_i)^2}{\sum_{i=1}^{n} (O_i - \bar{O})^2}
}\end{equation}

\begin{equation}\phantomsection\label{eq-kge}{
\text{KGE} = 1 - \sqrt{\left(\frac{\bar{S}}{\bar{O}}-1\right)^2 + (\alpha - 1)^2 + (\rho_p - 1)^2}
}\end{equation}

where \(\bar{O}\) and \(\bar{S}\) are the means of the observed and
simulated values, respectively, \(n\) is the number of observations,
\(\alpha\) is the ratio of simulated to observed standard deviation, and
\(\rho_p\) is the Pearson correlation the simulated and observed values.

Performance metrics were calculated over the full simulation period
using all available observations of canopy snow load and subcanopy SWE
at each site. To quantify sampling uncertainty associated with the
inclusion of potentially influential observations, a bootstrap
resampling procedure with replacement (10 000 bootstrap replicates) was
applied (Clark et al., 2021). For the canopy snow load evaluation,
snowfall events with observed snow load greater than 1 mm (n = 18) were
treated as resampling blocks. For the subcanopy SWE evaluation,
individual snow survey observations were used as resampling blocks, due
to their low temporal autocorrelation resulting from biweekly or monthly
sampling intervals.

\section{Results}\label{results}

\subsection{Snowpack Observations}\label{snowpack-observations}

Amongst the sites and years included in the study, accumulation of
snowfall below the canopy was less than cumulative snowfall observed in
open clearings adjacent to each forest transect
(Fig.~\ref{fig-sf-subcpy-swe}). At peak seasonal subcanopy SWE, Fortress
had the highest fraction of seasonal snowfall stored in the subcanopy
snowpack at 0.6, followed by Marmot and Wolf Creek which both had
similar fractions at 0.4, and Russell had the smallest fraction at
0.3---when averaged over all years. The variability across years was
highest at Wolf Creek ranging from 0.2 to 0.6
(Fig.~\ref{fig-frac-obs-swe}). Fortress and Marmot Creek also exhibited
substantial variability across years (Fig.~\ref{fig-frac-obs-swe}). In
contrast, Russell Creek showed the least variation; however, this
assessment is limited by observations from only two winter seasons
(Fig.~\ref{fig-frac-obs-swe}). Differences in partitioning of
intercepted snow by canopy snow unloading, melt/drip, and sublimation
contributed to the observed differences in subcanopy snow accumulation
relative to the open clearings across sites and years. At the
temperate-maritime Russell site, mid-winter melt events also contributed
to these observed differences. Section~\ref{sec-sf-partition} presents a
diagnosis of the dominant processes that partitioned snowfall across the
four different forest plots leading to the observed differences in
subcanopy snow accumulation.

\begin{figure}[H]

\centering{

\pandocbounded{\includegraphics[keepaspectratio]{figs/final/figure3.png}}

}

\caption{\label{fig-sf-subcpy-swe}Time series showing seasonal
cumulative snowfall (black lines) and subcanopy SWE from snow surveys
(blue dots). Note: snowfall was determined from observed total
precipitation to an open clearing for each site using the snowfall
fraction simulated in CRHM following Harder \& Pomeroy (2013).}

\end{figure}%

\begin{figure}[H]

\centering{

\includegraphics[width=0.75\linewidth,height=\textheight,keepaspectratio]{figs/final/figure4.png}

}

\caption{\label{fig-frac-obs-swe}Boxplots showing peak SWE as a fraction
of cumulative snowfall. Subcanopy SWE observations are from snow surveys
and snowfall was determined from observed total precipitation to an open
clearing for each site using the snowfall fraction simulated in CRHM
following Harder \& Pomeroy (2013). Note: the rectangle vertical extent
represents the interquartile range (25\textsuperscript{th} to
75\textsuperscript{th} percentile), the horizontal line within each box
indicates the median, and the whiskers extend to 1.5 times the
interquartile range. The white triangle denotes the mean across all
years.}

\end{figure}%

\subsection{Canopy Snow Load
Evaluation}\label{canopy-snow-load-evaluation}

Canopy snow load was overestimated by CP25 and underestimated by E10,
over the two observation periods between February 2007--February 2008
and December 2018--June 2019 at Marmot Creek
(Fig.~\ref{fig-cpy-load-ts}). The error in simulated canopy snow load
over the two periods was smaller for CP25 with a mean bias of -0.4 kg
m\textsuperscript{-2} compared to the E10 mean bias of 1 kg
m\textsuperscript{-2} (Table~\ref{tbl-cpy-load-mb}). The CP25 model also
produced higher NSE (0.83) and KGE (0.73) values compared to E10, which
had lower NSE (0.6) and KGE (0.33) values. CP25 demonstrated improved
accuracy across a range of snow loads, while E10 greatly underestimated
larger snow loads, as above the species specific canopy holding capacity
the E10 model decreases the fraction of new snow that is intercepted in
the canopy (Fig.~\ref{fig-cpy-load-ts}). Results from the bootstrap
analysis, which resampled different combinations of the 18 canopy
snowfall events at Marmot Creek, quantified the sampling uncertainty
associated with event selection. Across the resampled datasets, the CP25
model more accurately simulated canopy snow load, as indicated by its
higher KGE value (0.67) relative to the poorer performance of E10 (0.3)
(Fig.~\ref{fig-cpy-load-boot}). The E10 model consistently
underestimated canopy snow load, as shown by a positive mean bias across
all event combinations. In contrast, CP25 exhibited a smaller negative
mean bias that was closer to zero, although it also underestimated snow
load for some event combinations, as indicated by the positive upper
bound of the 95\% confidence interval.

For five events between May and June 2019, characterised by mean air
temperatures above 0°C with mixed snowfall and rainfall, canopy snow
load was underestimated by both CP25 and E10 (see Supporting Information
for a subset of Fig.~\ref{fig-cpy-load-ts} for just these events). Two
precipitation events (2019-05-30 and 2019-06-14) were simulated by CRHM
entirely as rain. During these events, observed canopy loads reached
2.6--3.0 kg m\textsuperscript{-2}, indicative of snow interception,
whereas simulated loads were \textless{} 0.5 kg m\textsuperscript{-2},
close to the liquid storage capacity used in both CP25 and E10. For the
remaining three events with mixed snow/rain on 2019-05-15, 2019-05-23,
and 2019-06-06 CP25 underestimated peak canopy snow load by -18.1\% to
-84.4\% and E10 by -54.7\% to -69.4\%. The underestimation of simulated
canopy load by both models over these events is attributed to the liquid
storage capacities of canopy snow and vegetation elements, not
representing differing unloading rates with increased cohesion/adhesion
of snow to the canopy near the melting point, instrument uncertainty in
the observations, and errors in the precipitation phase
parameterisation.

Over the 2018--2019 period the fraction of the year that the canopy was
loaded with \textgreater2 kg m\textsuperscript{-2} of snow, was found to
be 0.26, compared to simulations of 0.29 and 0.16 by the CP25 and E10
models respectively. A threshold of 2 kg m\textsuperscript{-2} was
selected based on observations by Pomeroy \& Dion (1996) who found
minimal influence of canopy snow load on above canopy albedo for loads
less than 1.6 kg m\textsuperscript{-2}. The underestimate of canopy
intercepted snow duration by the E10 model of around -40.1\% results
from the underestimation of canopy load which depleted the canopy of
snow earlier than the observations. The CP25 model slightly
overestimated the intercepted snow duration by 10.2\%, resulting from
the overestimates of canopy snow load by CP25
(Fig.~\ref{fig-cpy-load-ts}) which stored canopy snow loads above
\textgreater2 kg m\textsuperscript{-2} longer than the observations.

\begin{figure}[H]

\centering{

\pandocbounded{\includegraphics[keepaspectratio]{figs/final/figure5.png}}

}

\caption{\label{fig-cpy-load-ts}Timeseries of observed and simulated
(CP25 and E10) canopy snow load at Marmot Creek for two periods February
2007 to February 2008 (top) and December 2018 to June 2019 (bottom).}

\end{figure}%

\begin{figure}[H]

\centering{

\pandocbounded{\includegraphics[keepaspectratio]{figs/final/figure6.png}}

}

\caption{\label{fig-cpy-load-boot}Error statistics derived from
bootstrap resampling of differing combinations of canopy snow load
events (10 000 replicates) at Marmot Creek between December 2018 and
June 2019. Points indicate the mean metric, and error bars show the 95\%
confidence intervals estimated across all resampled event combinations.}

\end{figure}%

\pagebreak

\begin{longtable}[]{@{}
  >{\raggedright\arraybackslash}p{(\linewidth - 12\tabcolsep) * \real{0.1000}}
  >{\raggedright\arraybackslash}p{(\linewidth - 12\tabcolsep) * \real{0.2000}}
  >{\raggedleft\arraybackslash}p{(\linewidth - 12\tabcolsep) * \real{0.1000}}
  >{\raggedleft\arraybackslash}p{(\linewidth - 12\tabcolsep) * \real{0.1000}}
  >{\raggedleft\arraybackslash}p{(\linewidth - 12\tabcolsep) * \real{0.1000}}
  >{\raggedleft\arraybackslash}p{(\linewidth - 12\tabcolsep) * \real{0.1000}}
  >{\raggedleft\arraybackslash}p{(\linewidth - 12\tabcolsep) * \real{0.1000}}@{}}

\caption{\label{tbl-cpy-load-mb}Mean bias (MB), root mean squared error
(RMSE), Nash-Sutcliffe efficiency (NSE), and Kling-Gupta efficiency
(KGE) determined from time-series simulations of canopy snow load for
the two models CP25 and E10 at Marmot Creek. The final column (n) shows
the count of observations used to compute the statistics.}

\tabularnewline

\toprule\noalign{}
\begin{minipage}[b]{\linewidth}\raggedright
Model
\end{minipage} & \begin{minipage}[b]{\linewidth}\raggedright
Year
\end{minipage} & \begin{minipage}[b]{\linewidth}\raggedleft
MB
\end{minipage} & \begin{minipage}[b]{\linewidth}\raggedleft
RMSE
\end{minipage} & \begin{minipage}[b]{\linewidth}\raggedleft
NSE
\end{minipage} & \begin{minipage}[b]{\linewidth}\raggedleft
KGE
\end{minipage} & \begin{minipage}[b]{\linewidth}\raggedleft
n
\end{minipage} \\
\midrule\noalign{}
\endhead
\bottomrule\noalign{}
\endlastfoot
CP25 & 2007-2008 & -1.93 & 3.26 & 0.50 & 0.60 & 11 \\
E10 & 2007-2008 & 3.82 & 5.30 & -0.33 & 0.35 & 11 \\
CP25 & 2018-2019 & -0.37 & 1.58 & 0.83 & 0.73 & 3838 \\
E10 & 2018-2019 & 1.03 & 2.39 & 0.60 & 0.33 & 3838 \\
CP25 & All & -0.37 & 1.59 & 0.83 & 0.73 & 3849 \\
E10 & All & 1.04 & 2.40 & 0.60 & 0.33 & 3849 \\

\end{longtable}

\subsection{Subcanopy Snowpack
Evaluation}\label{subcanopy-snowpack-evaluation}

Over all years and sites, the CP25 model had a lower mean bias of -1.68
kg m\textsuperscript{-2} compared to E10's mean bias of -24.5 kg
m\textsuperscript{-2} in representing subcanopy SWE measurements
(Table~\ref{tbl-swe-mb}). Model errors were lower for the three colder
climate sites (i.e., Fortress, Marmot, and Wolf Creek), where CP25
underestimated SWE (MB = 2.87 kg m\textsuperscript{-2}) and E10
overestimated SWE (MB = -9.99 kg m\textsuperscript{-2}). Although E10
had a marginally reduced mean bias at Fortress compared to CP25, the
RMSE, NSE, and KGE values were improved across all sites for CP25
reflecting improved accuracy of the new model (Table~\ref{tbl-swe-mb}).
Three years contributed to the higher RMSE and lower NSE and KGE values
by E10 at Marmot where simulated peak SWE was greatly overestimated by
over 50 kg m\textsuperscript{-2} (nearly 100\% of observed SWE) for the
water years 2011 and 2012. In contrast, E10 had a very large
underestimation in subcanopy SWE at Marmot for the water year 2019. At
Wolf Creek, E10 also had deviations of \textasciitilde30 kg
m\textsuperscript{-2} (\textasciitilde100\% greater than observed SWE)
from peak SWE for 2016 and 2017. At Russell Creek, errors were lower for
both CP25 and E10 during the first winter season than the second. This
difference may be explained by a higher frequency of cold snowfall
events during the first year, when air temperatures frequently dropped
below −10 °C, compared with the second year when temperatures did not
fall below −8 °C. The colder conditions in the first year were more
consistent with those at Fortress Mountain, where the CP25
parameterisations were developed. In contrast, the relatively warmer
second winter could have promoted stronger melt-freeze processes,
resulting in more cohesive canopy snow, greater canopy retention, and
reduced partitioning into solid snow unloading. Although the intent of
this evaluation is to assess the performance of each model in simulating
subcanopy SWE, uncertainties in model forcing and physical parameters
(i.e., canopy coverage, LAI) may also contribute to systematic biases in
the evaluation. Some of the increased model error at Russell during the
2008 water year may reflect the use of total precipitation data from a
nearby highway station rather than on-site measurements.

Bootstrap resampling revealed clear differences in model performance in
simulating subcanopy snow water equivalent (SWE) across the evaluated
sites (Fig.~\ref{fig-swe-boot}, Fig.~\ref{fig-swe-boot-rus}). The CP25
model had consistently lower RMSE, KGE, and NSE mean values, as well as
a smaller 95\% confidence interval, across all four sites. The mean bias
statistic differed less between the two models with E10 showing a
slightly lower mean bias at Fortress, versus lower mean bias for CP25 at
Marmot. Still, the range in mean biases across the differing event
combinations for CP25 was smaller showing more stable performance.
Across the three cold climate sites (Fortress, Marmot, and Wolf Creek)
the greatest deviation in the error statistics between the two models
was observed at Marmot Creek, with improved accuracy for the CP25 model
and less difference in performance at Fortress and Wolf Creek. Both CP25
and E10 had higher errors at Russell Creek compared to the colder
climate sites, though CP25 still had substantially greater accuracy
compared to CP25 due to a better representation of canopy snow ablation
processes (Fig.~\ref{fig-swe-boot-rus}).

\pagebreak

\begin{longtable}[]{@{}
  >{\raggedright\arraybackslash}p{(\linewidth - 12\tabcolsep) * \real{0.1000}}
  >{\raggedright\arraybackslash}p{(\linewidth - 12\tabcolsep) * \real{0.3000}}
  >{\raggedleft\arraybackslash}p{(\linewidth - 12\tabcolsep) * \real{0.1000}}
  >{\raggedleft\arraybackslash}p{(\linewidth - 12\tabcolsep) * \real{0.1000}}
  >{\raggedleft\arraybackslash}p{(\linewidth - 12\tabcolsep) * \real{0.1000}}
  >{\raggedleft\arraybackslash}p{(\linewidth - 12\tabcolsep) * \real{0.1000}}
  >{\raggedleft\arraybackslash}p{(\linewidth - 12\tabcolsep) * \real{0.1000}}@{}}

\caption{\label{tbl-swe-mb}Mean bias (MB), root mean squared error
(RMSE), Nash-Sutcliffe efficiency (NSE), and Kling-Gupta efficiency
(KGE) determined from time-series simulations of snow water equivalent
for the two canopy snow models at each of the four sites. The final
column (n) shows the count of observations used to compute the
statistics.}

\tabularnewline

\toprule\noalign{}
\begin{minipage}[b]{\linewidth}\raggedright
Model
\end{minipage} & \begin{minipage}[b]{\linewidth}\raggedright
Station
\end{minipage} & \begin{minipage}[b]{\linewidth}\raggedleft
MB
\end{minipage} & \begin{minipage}[b]{\linewidth}\raggedleft
RMSE
\end{minipage} & \begin{minipage}[b]{\linewidth}\raggedleft
NSE
\end{minipage} & \begin{minipage}[b]{\linewidth}\raggedleft
KGE
\end{minipage} & \begin{minipage}[b]{\linewidth}\raggedleft
n
\end{minipage} \\
\midrule\noalign{}
\endhead
\bottomrule\noalign{}
\endlastfoot
CP25 & All Station Mean & -1.68 & 45.8 & 0.86 & 0.92 & 282 \\
E10 & All Station Mean & -24.54 & 110.3 & 0.20 & 0.55 & 282 \\
CP25 & Fortress - Powerline Forest & 5.10 & 43.8 & 0.89 & 0.94 & 109 \\
E10 & Fortress - Powerline Forest & -4.58 & 52.2 & 0.84 & 0.90 & 109 \\
CP25 & Marmot - Upper Forest & 2.29 & 20.9 & 0.69 & 0.79 & 124 \\
E10 & Marmot - Upper Forest & -5.02 & 30.6 & 0.33 & 0.62 & 124 \\
CP25 & Russell - Old Growth & -127.40 & 162.7 & -7.24 & -0.95 & 12 \\
E10 & Russell - Old Growth & -465.37 & 500.0 & -76.80 & -3.63 & 12 \\
CP25 & Wolf Creek - Forest & 5.78 & 17.4 & 0.70 & 0.75 & 37 \\
E10 & Wolf Creek - Forest & -5.81 & 21.9 & 0.52 & 0.71 & 37 \\

\end{longtable}

The evolution of season subcanopy SWE was generally represented well by
both models at Fortress, Marmot, and Wolf Creek (Fig.~\ref{fig-swe-ts}).
However, E10 failed to simulate the timing of SWE accumulation and
ablation well for 2011, 2012, and 2019 at Marmot. The largest deviation
in simulated seasonal SWE occurred at Russell Creek for E10 where
subcanopy snow accumulation was simulated at a much higher rate compared
to that observed and estimated by CP25 over the two years that were
simulated. At Wolf Creek, CP25 had a delay in simulating the initial
accumulation of subcanopy SWE for water years 2017, 2018, and 2019. The
lower snowfall rate at Wolf Creek and higher interception rate for CP25
compared to E10 led to throughfall and unloading rates that were smaller
than the snowpack initiation threshold employed in Snobal, causing
simulated SWE below this threshold to melt immediately. In contrast, E10
intercepted less snow and had higher unloading rates than CP25 leading
to higher initial accumulation for these three years at Wolf Creek.

\begin{figure}[H]

\centering{

\pandocbounded{\includegraphics[keepaspectratio]{figs/final/figure7.png}}

}

\caption{\label{fig-swe-ts}Timeseries of observed and simulated (CP25
and E10) forest snow water equivalent at each station.}

\end{figure}%

\begin{figure}[H]

\centering{

\pandocbounded{\includegraphics[keepaspectratio]{figs/final/figure8.png}}

}

\caption{\label{fig-swe-boot}Error statistics from bootstrap resampling
of differing combinations of subcanopy SWE measurements (10 000
replicates) at Fortress Mountain, Marmot Creek, and Wolf Creek. Russell
Creek is shown in Fig.~\ref{fig-swe-boot-rus} due to the differing
magnitude of error. Points indicate the mean metric and error bars show
the 95\% confidence intervals estimated across all resampled event
combinations.}

\end{figure}%

\begin{figure}[H]

\centering{

\pandocbounded{\includegraphics[keepaspectratio]{figs/final/figure9.png}}

}

\caption{\label{fig-swe-boot-rus}Error statistics from bootstrap
resampling of differing combinations of snow surveys (10 000 replicates)
at Russell Creek. Points indicate the mean metric and error bars show
the 95\% confidence intervals estimated across all resampled event
combinations.}

\end{figure}%

\pagebreak

\subsection{Simulated Canopy Snow
Load}\label{simulated-canopy-snow-load}

Over all years and sites, CP25 predicted consistently higher canopy snow
loads compared to the E10 model (Fig.~\ref{fig-cpy-swe}). This was due
to E10's increase in throughfall with increasing antecedent snow load as
well as the unloading rate as a function of snow load and ice-bulb
temperature which led to increased throughfall and unloading for E10 and
less snow residing in the canopy compared to CP25. Some snowfall events
had similar initial accumulation of snow in the canopy between the two
models up until the E10 species snow load capacity was reached (see
events in Jan and Feb at Fortress, Marmot, and Wolf Creek in the
Supporting Information).

\begin{figure}[H]

\centering{

\pandocbounded{\includegraphics[keepaspectratio]{figs/final/figure10.png}}

}

\caption{\label{fig-cpy-swe}Timeseries of simulated canopy load for CP25
and E10 at each station for the full simulation period.}

\end{figure}%

In addition to the large difference in canopy snow load predicted by
CP25 and E10 across all four sites, the duration that the canopies of
each site were simulated to have more than 2 kg m\textsuperscript{-2} of
snow varied between the four sites and two models
(Fig.~\ref{fig-frac-cpy-load-th}). Across all four sites, canopy snow
loads greater than 2 kg m\textsuperscript{-2} were observed to persist
longer for CP25 compared to E10 (Fig.~\ref{fig-frac-cpy-load-th}). The
largest differences between the two models occurred at Russell and
Fortress, where snow loads exceeding this threshold persisted 131\% and
75\% longer in CP25 than in E10. At Marmot and Wolf Creek, the relative
differences were smaller---61\% and 48\%---due to lower snowfall and
canopy snow loads at these sites, resulting in slightly higher
interception efficiencies for the E10 model. A sensitivity analysis of
the canopy snow load threshold revealed similar model behaviour when
using a threshold of 1.6 kg m\textsuperscript{-2}, based on observations
from Pomeroy \& Dion (1996), compared to the 2 kg m\textsuperscript{-2}
threshold. Differences between the two models increased for larger
canopy snow load thresholds and were slightly reduced for smaller
thresholds.

\begin{figure}[H]

\centering{

\pandocbounded{\includegraphics[keepaspectratio]{figs/final/figure11.png}}

}

\caption{\label{fig-frac-cpy-load-th}Boxplots showing the annual
fraction of time when simulated canopy snow load is greater than 2 kg
m\textsuperscript{-2} by the CP25 and E10 models.}

\end{figure}%

\subsection{Snowfall Partitioning and
Disposition}\label{sec-sf-partition}

A greater fraction of annual snowfall was sublimated by CP25 compared to
E10 for all four sites across all years (Fig.~\ref{fig-frac-cpy-proc}).
Lower interception efficiency combined with higher average rates of
unloading for the E10 model led to more snowfall being partitioned
towards the ground compared to the CP25 model
(Fig.~\ref{fig-frac-cpy-proc}) leading to reduced sublimation of canopy
snow and underprediction of canopy SWE (Table~\ref{tbl-cpy-load-mb}) and
overprediction of subcanopy SWE accumulation (Table~\ref{tbl-swe-mb}).
The underprediction of subcanopy SWE by the CP25 model at Fortress,
Marmot, and Wolf Creek (Table~\ref{tbl-swe-mb}) may have been due to an
overestimate of sublimation and/or canopy snowmelt rates leading to less
snowfall being partitioned to the ground as solid snow. The difference
in the annual fraction of snowfall that was sublimated between the CP25
and E10 models was most prevalent at Marmot
(Fig.~\ref{fig-frac-cpy-proc}). Factors that may contribute to this
large deviation include the lower unloading rates observed for CP25 at
this site (Fig.~\ref{fig-frac-cpy-proc}) compared to E10 resulting from
the lower wind speed and air temperatures at this site
(Fig.~\ref{fig-met-normals}), reducing the unloading and canopy snowmelt
rates, thus increasing the amount of canopy snow subject to sublimation
for CP25. At Marmot Creek, the monthly air temperature normals between
April and June---when this site receives most of its snowfall
(Fig.~\ref{fig-met-normals})---are also largely within the E10 ice-bulb
temperature unloading range (-3°C to 6°C) for initiating unloading due
to warming. Relatively similar fractions of annual snowfall were
sublimated by the two models at Fortress and Wolf Creek
(Fig.~\ref{fig-frac-cpy-proc}) due to the general agreement in the two
sublimation parameterisation by the two models combined with the similar
fraction of snow partitioned towards the ground
(Fig.~\ref{fig-frac-cpy-proc}). The drip/melt fraction simulated by E10
was near zero for all four sites, while CP25 had fractions ranging
between 0.1 to 0.38 (Fig.~\ref{fig-frac-cpy-proc}). As a result, the E10
model incorrectly partitioned canopy snow ablation entirely as solid
snow unloading for melt events and also contributed to the
overestimation of subcanopy SWE (Table~\ref{tbl-swe-mb}). Process
observations of initial interception, unloading, drip, and sublimation
were evaluated in Cebulski \& Pomeroy (2025b) and Cebulski \& Pomeroy
(2025c) at Fortress Mountain; however, these observations were not
available at the other sites and thus the simulations of snowfall
partitioning could not be directly evaluated.

The results from the new CP25 model show that at Marmot and Wolf Creek,
two cold, low-snowfall sites with calm winds, snowfall was primarily
partitioned into sublimation of intercepted snow
(Fig.~\ref{fig-net-cpy-proc}). At Russell Creek, warm temperatures, high
humidities and calm winds resulted in snowfall primarily partitioning
into unloading and canopy snow drip. Cold air temperatures and increased
wind speeds at Fortress increased the relative contribution of
unloading, compared to melt and sublimation. The difference in
throughfall relative to the mean water year snowfall across the four
sites results from differences in the calculated snow-leaf contact area
used in the initial snow interception calculation, which is a function
of the site canopy coverage as well as hydrometeor trajectory zenith
angle (departure in degrees from a vertical plane) as defined in
Cebulski \& Pomeroy (2025b).

\begin{figure}[H]

\centering{

\pandocbounded{\includegraphics[keepaspectratio]{figs/final/figure12.png}}

}

\caption{\label{fig-frac-cpy-proc}Boxplots showing the distribution of
the fraction of total snowfall that was sublimated out of the canopy
(top row), reached the subcanopy via unloading and/or throughfall
(middle row), and drip from melting intercepted snow (bottom row) at
each station for each water year (year ranges are shown in
Table~\ref{tbl-site-meta}). Note: the rectangle vertical extent
represents the interquartile range (25\textsuperscript{th} to
75\textsuperscript{th} percentile), the horizontal line within each box
indicates the median, and the whiskers extend to 1.5 times the
interquartile range. Circular points beyond the whiskers represent
outliers.}

\end{figure}%

\begin{figure}[H]

\centering{

\pandocbounded{\includegraphics[keepaspectratio]{figs/final/figure13.png}}

}

\caption{\label{fig-net-cpy-proc}Bar chart of the mean water-year totals
of snowfall, throughfall, unloading, canopy snow sublimation, and drip
from melting intercepted snow at each station, averaged across all
years. The year ranges for each station are given in
Table~\ref{tbl-site-meta}.}

\end{figure}%

\section{Discussion}\label{discussion}

\subsection{Model Performance}\label{model-performance}

New parameterisations of the canopy snow energy and mass
balance---supported by advances in process understanding (Cebulski \&
Pomeroy, 2025b, 2025c; Lundquist et al., 2021; Staines \& Pomeroy,
2023)---were evaluated for their ability to simulate SWE stored within
and below the canopy. Inclusion of both dry- and melt-induced unloading
is supported by observations (Cebulski \& Pomeroy, 2025c; Ellis et al.,
2010; Floyd, 2012; Lumbrazo et al., 2022; Roesch et al., 2001), with
mechanisms such as bond weakening, lubrication during melt, and wind
shear reinforcing their physical basis. An energy balance-based canopy
snowmelt routine---recommended by many studies (Andreadis et al., 2009;
Cebulski \& Pomeroy, 2025c; Lumbrazo et al., 2022; Lundquist et al.,
2021; Storck et al., 2002) also contributed to improved accuracy of
simulated canopy and subcanopy SWE. The higher canopy snow loads
simulated by the new model are consistent with empirical observations in
this study and others (Calder, 1990; Cebulski \& Pomeroy, 2025b;
Hedstrom \& Pomeroy, 1998; Storck et al., 2002; Watanabe \& Ozeki,
1964), which demonstrate a linear increase in interception with snowfall
and limited evidence of a maximum capacity. Specifically, simulated
canopy snow loads reaching close to 50 kg m\textsuperscript{-2} are
consistent with observations in coastal environments by Storck et al.
(2002) and Floyd (2012). By calculating throughfall as a function of
canopy density (Cebulski \& Pomeroy, 2025b; Staines \& Pomeroy, 2023)
and combining this with a comprehensive canopy snow ablation routine
(Cebulski \& Pomeroy, 2025c; Lundquist et al., 2021), results from
Cebulski \& Pomeroy (2025b) and Table~\ref{tbl-cpy-load-mb} showed
canopy snow loads simulated using the new initial interception
parameterisation were more representative of observations. Calculating
throughfall as a function of antecedent snow load, as implemented in the
E10 model, combined with unloading rates parameterised by ice-bulb
temperature and/or canopy snow load, resulted in underestimation of both
the amount and duration of snow intercepted in the canopy. While the new
model reduced errors in simulated snow load
(Fig.~\ref{fig-cpy-load-boot}), canopy snow loads were generally
overestimated compared to observations at Marmot Creek for cold snow
events and suggests dry-snow unloading rates may be higher in this
forest. For mixed rain/snow events canopy snow load was underestimated
by the new model and E10 due to underestimates in the liquid water
storage capacity and/or overestimates in canopy snow ablation during
these events. The liquid water storage capacities for both models were
computed using Equation 2 from Cebulski \& Pomeroy (2025c), suggesting
that the coefficients in this equation may require adjustment for this
forest site. Rainfall interception studies have demonstrated that the
canopy water storage capacity varies across forest age classes (Pypker
et al., 2005) and species (Xiao \& McPherson, 2016). However, further
research is needed to understand how liquid water storage capacity
differs amongst species and age classes, and how these differences
manifest under varying levels of intercepted snow. Moreover, the
unloading rates over these warmer events may be lower compared to the
coefficients developed at Fortress Mountain in Cebulski \& Pomeroy
(2025c), potentially due to increases in cohesion and adhesion of snow
to the canopy due to increases in liquid water content and/or
melt-freeze processes within the canopy. Lumbrazo et al. (2022) also
showed that parameters for unloading are site specific and may suggest
the initial interception parameters or model process conceptions are as
well. However, the large improvement in both the uncalibrated simulation
of canopy snow load and intercepted snow duration by the new model
provides some evidence of increased transferability across differing
forest canopies. Despite some discrepancies in simulating canopy snow
load by the new model at Marmot Creek, the new model more closely
approximated the duration the canopy was loaded with more than 2 kg
m\textsuperscript{-2} of snow---based on the threshold identified by
Pomeroy \& Dion (1996) to be sufficient to impact above canopy
albedo---reducing error compared to an existing model by a factor of
four.

The improved representation of canopy snow ablation as demonstrated by
Cebulski \& Pomeroy (2025c) at Fortress Mountain for individual snowfall
events under warm conditions likely contributed to significant reduction
in simulated subcanopy SWE errors at the temperate-maritime Russell
site. In contrast, the E10 ice-bulb temperature-based melt threshold
(\textgreater{} 6 °C) was infrequently reached and caused ablation to
occur mainly as solid snow unloading rather than drip. Overestimates of
canopy snow unloading during both dry-snow and melt-driven conditions by
the E10 model led to overpredicted subcanopy SWE, as less intercepted
snow was exposed to sublimation or melt (Table~\ref{tbl-swe-mb}). Low
error in the new model simulated subcanopy SWE at the cold sites (i.e.,
Fortress, Marmot, and Wolf Creek) suggests that the dry snow unloading
parameterisation is transferable across sites and is consistent with
observations by Lumbrazo et al. (2022), where a wind-driven unloading
parameterisation had good accuracy across differing cold-climate sites.
Despite relatively similar mean biases in subcanopy SWE simulations
across the cold-climate sites (Table~\ref{tbl-swe-mb}), the new model
had reduced RMSE and higher KGE and NSE values. This resulted from an
improved representation of snow interception and canopy snow ablation
processes which better captured the variation in snowfall partitioning
over years with differing snowfall amounts and meteorology. Both the new
model and E10 model performed worst at the temperate-maritime Russell
site---although the new model showed large improvements over E10---where
melt/drip and melt-induced unloading processes dominated. Remaining
error at Russell may reflect unrepresented processes, such as ice
accretion from canopy snow melt-freeze cycles and rime-ice deposition as
observed in other maritime forests (Lumbrazo et al., 2022), which can
increase canopy loads, lower unloading rates, and favour partitioning to
liquid water. Additional uncertainties stem from simplifications in the
canopy energy balance (e.g., radiation transmittance, longwave emission,
and turbulent fluxes), as well as parameterisations of interception and
unloading (Cebulski \& Pomeroy, 2025b, 2025c). These issues are likely
amplified at Russell, where frequent air temperatures near the melting
point cause large changes in enthalpy which increase phase change
sensitivity to energy balance formulations, compared to the colder sites
with more stable enthalpy conditions. Some errors in subcanopy SWE
simulations are also attributed to limitations in the subcanopy snowpack
energy and mass balance routine (i.e., Snobal). For instance, a minimum
accumulation threshold caused discrepancies in subcanopy SWE simulations
at Wolf Creek.

The hydrometeor fall velocity and wind shear stress used in the initial
interception and canopy snow unloading parameterisations, respectively,
were estimated using simplified approximations that introduced error
into the new model's results at all four sites. For instance, a fixed
hydrometeor fall velocity was used at all four sites based on empirical
observations from Cebulski \& Pomeroy (2025b) and Isyumov (1971). This
approximation increases uncertainty in the initial interception
calculation due to the associated sensitivity of wind speed and
hydrometeor fall velocity with snow-leaf contact area. Furthermore,
midcanopy shear stress was approximated using an empirical relationship
with midcanopy wind speed developed in Cebulski \& Pomeroy (2025c). The
use of more physically based formulations that account for variations in
shear stress resulting from differences in air density and/or
within-canopy wind flow (Stull, 2017) could further improve estimates of
shear stress and, consequently, unloading.

The new model simulated much longer periods of canopy load greater than
2 kg m\textsuperscript{-2} and improved accuracy at Marmot Creek
compared to an existing approach (Table~\ref{tbl-cpy-load-mb}).
Sensitivity to the canopy snow load threshold was observed, with greater
intermodel discrepancies for larger threshold values and slightly
smaller differences for lower thresholds. Greater model differences were
observed for the larger thresholds as the new model predicted much
larger canopy snow loads across all sites and years
(Fig.~\ref{fig-cpy-swe}). This improved simulation of canopy snow load
has implications for the representation of above-canopy albedo in land
surface models which have previously shown poor performance with
existing canopy snow ablation models (Thackeray et al., 2014; Wang et
al., 2016). Moreover, E10's performance in simulating subcanopy SWE
varied substantially between years; it overestimated SWE over most
years, but underestimated SWE in low-snowfall years when reduced
unloading rates allowed more sublimation (e.g., Marmot in 2019;
Fig.~\ref{fig-swe-ts}). These results show that recent improvements in
process parameterisations---particularly the treatment of canopy load,
ablation, and unloading---yield measurable improvements in simulating
canopy and subcanopy SWE across diverse climates and forest types.

\subsection{Influence of Climate on Snowfall Partitioning and
Disposition}\label{influence-of-climate-on-snowfall-partitioning-and-disposition}

Although Russell and Fortress received similar amounts of snowfall
(Fig.~\ref{fig-sf-subcpy-swe}), subcanopy accumulation was 50\% lower at
Russell due to greater canopy snowmelt: \textasciitilde40\% of annual
snowfall melted and 20\% sublimated in the canopy, compared to
\textasciitilde35\% combined melt and sublimation of intercepted snow at
Fortress. The low cold content of the maritime snowpack at Russell
limited refreezing of canopy snow meltwater, and relatively warm air
temperatures (Fig.~\ref{fig-met-normals}) promoted melt of the subcanopy
snowpack throughout the season, reducing subcanopy snowpack accumulation
(Fig.~\ref{fig-swe-ts}). Consequently, the prevalence of snowmelt both
in the canopy and on the ground at Russell led to the lowest fraction of
seasonal snowfall stored as subcanopy snow accumulation at peak SWE
(Fig.~\ref{fig-frac-obs-swe}). Canopy snow sublimation was also lowest
at Russell, as most ablation occurred via melt and solid snow unloading
(Fig.~\ref{fig-frac-cpy-proc}, Fig.~\ref{fig-net-cpy-proc}). Frequent
mid-winter canopy snowmelt also provided a steady year-round input of
drip to the subcanopy, which recharges soil moisture and groundwater,
and promotes runoff generation (Barnhart et al., 2016; Groff \& Pomeroy,
2025; Hayashi, 2020). At Fortress, colder conditions, higher seasonal
snowfall, and greater wind exposure increased unloading rates and
limited canopy snow sublimation (Fig.~\ref{fig-frac-cpy-proc},
Fig.~\ref{fig-net-cpy-proc}). Combined with low canopy snow drip
(Fig.~\ref{fig-net-cpy-proc}), these processes at Fortress yielded the
highest subcanopy accumulation (0.6) across all four sites
(Fig.~\ref{fig-frac-obs-swe}). Marmot and Wolf Creek experienced cooler,
drier conditions with calm winds and lower seasonal snowfall, leading to
reduced unloading and increased the fraction and amount of snowfall that
was sublimated from the canopy (Fig.~\ref{fig-frac-cpy-proc},
Fig.~\ref{fig-net-cpy-proc}). This, coupled with canopy snowmelt playing
a minor role in partitioning intercepted snow meant that only
\textasciitilde0.4 of snowfall accumulated as subcanopy snowpack at peak
SWE (Fig.~\ref{fig-frac-obs-swe}). Simulated sublimation at Marmot and
Wolf Creek (\textasciitilde50\% of seasonal snowfall;
Fig.~\ref{fig-frac-cpy-proc}) exceeded the upper range of global
estimates (25--45\%) reported by Essery et al. (2003) but is consistent
with observations by Pomeroy \& Gray (1995), Pomeroy et al. (1998), and
Ellis et al. (2010).

These results also extend earlier discussions by Pomeroy \& Gray (1995)
on the fraction of seasonal snowfall stored in forests, by demonstrating
how the partitioning and disposition of intercepted snow varies across
differing climates and forests (Fig.~\ref{fig-net-cpy-proc}). At Russell
Creek, a temperate forest, snowfall was predominantly partitioned into
unloading and canopy snow drip, with minimal sublimation. At Marmot and
Wolf Creek two cold-regions forests, characterised by low seasonal
snowfall and calm winds, snowfall was primarily returned to the
atmosphere via sublimation of intercepted snow. Fortress, a cold,
wind-exposed forest with higher snowfall, was dominated by unloading.
The agreement in the new model and E10 in partitioning snowfall between
sublimation and unloading at Fortress and Wolf Creek arises because the
E10 routine unloads a similar fraction of annual snowfall as the new
model despite having different parameterisations for unloading and
sublimation. For the two sublimation schemes, E10 omits longwave
radiation from the canopy snow sublimation energy balance (Pomeroy et
al., 1998), while parameterisation used in the new model includes
longwave radiation (Essery et al., 2003). This consistency between the
two sublimation parameterisations was also observed by Cebulski \&
Pomeroy (2025c) during several sublimation-dominated ablation events.

\subsection{Influence of Tree Species on Snowfall
Partitioning}\label{influence-of-tree-species-on-snowfall-partitioning}

The species of needleleaf forest overlying each site also influenced
subcanopy SWE accumulation. However, limited research has been conducted
on how branch elasticity, needle composition, and crown structure
between differing species and ages affect the interception and ablation
of snow. The Marmot and Wolf Creek sites, both primarily composed of
spruce trees (Table~\ref{tbl-site-meta}) and exposed to similarly cool
climates (Fig.~\ref{fig-met-normals}), showed comparable fractions of
snowfall stored in the subcanopy (Fig.~\ref{fig-frac-cpy-load-th}) and
similar canopy melt and sublimation losses
(Fig.~\ref{fig-frac-cpy-proc}). In contrast, Fortress and Russell were
both primarily fir-dominated forests, yet diverged strongly due to
climate differences (Fig.~\ref{fig-met-normals}) and vegetation
structure. Even though Russell had approximately 20\% more canopy cover
than Fortress, it accumulated only about half as much snowfall beneath
the canopy. Tree size is a likely factor, large 50 m tall fir trees at
Russell can intercept far more snow than the smaller 10 m fir trees at
Fortress. Coupled with low wind speeds, this increased exposure of
intercepted snow to canopy energy fluxes that favoured melt and drip at
Russell Creek. The larger, more supportive branches combined with the
differing climatic regime and potential for ice-deposition may also have
slowed unloading, contributing to the new model's overestimate of
subcanopy SWE at Russell. These results support earlier theory posed by
Satterlund \& Haupt (1970) that vegetation structure, density, and
climate exert stronger control on forest-snow partitioning than tree
species alone. Moreover, Schmidt \& Pomeroy (1990) showed that branch
temperature rather than species type had the greatest influence on the
modulus of elasticity of needleleaf branches and their snow holding
capacity. However, other forest types, such as broadleaf deciduous
(Huerta et al., 2019), cedar or hemlock stands, or younger trees, with
less supportive leaves and branches, may have a stronger influence on
canopy-snow processes than the relatively subtle species differences
observed amongst needleleaf sites in this study.

\section{Conclusions}\label{conclusions}

Recent advances in canopy snow energy and mass balance parameterisations
were shown to reduce errors in simulated SWE, both within and below the
canopy, across forests spanning a range of climates and canopy
structures. The new model also provided a more robust diagnosis of the
processes that govern how intercepted snowfall is partitioned between
sublimation to the atmosphere, storage in the canopy, drip, and
unloading to the forest floor. Building on recent developments, the new
approach calculates initial snow interception as a function of canopy
density and hydrometeor trajectory angle. Canopy snow ablation is
represented by melt- and wind-driven unloading, energy balance-based
melt, and sublimation processes that vary with canopy snow load. The new
parameterisations improved simulations of the quantity and duration that
snow was intercepted in the canopy compared to an existing approach.
Simulation of subcanopy SWE improved across all four sites and is
attributed to more accurately representing the canopy snow energy and
mass balance in the new model. This was most apparent in a
temperate-maritime environment, where errors in subcanopy SWE were a
factor of four lower---attributed to a more accurate representation of
initial interception, canopy snowmelt, and unloading. These results
highlight the robustness of physically based parameterisations over a
wide range of climatic conditions, which is particularly important given
that continued warming may reduce the applicability of empirically
derived routines.

Process diagnosis conducted with the new snow interception and canopy
snow ablation parameterisations highlighted the role of needleleaf
canopies in partitioning intercepted snowfall. The greatest influence
was observed in the temperate-maritime forest, where increased energy
fluxes to intercepted snow caused nearly half of seasonal snowfall to
melt within the canopy, producing the lowest subcanopy SWE fraction of
snowfall and a steady contribution of meltwater throughout the winter.
Despite the large influence of snowfall partitioning by vegetation on
the subcanopy snowpack, sublimation losses were relatively small
(\textasciitilde20\% of seasonal snowfall). At two cold continental
sites with lower annual snowfall, reduced unloading led to a larger
proportion of intercepted snow to be retained in the canopy over longer
durations and returned to the atmosphere via sublimation
(\textasciitilde40\%). In contrast, the cold wind-exposed site with
higher snowfall exhibited greater unloading and shorter canopy residence
times, which limited sublimation losses relative to the other cold
sites. Overall, the partitioning and disposition of intercepted snow was
observed to strongly vary across climates: temperate-maritime forests
were dominated by canopy snow drip, cold, wind-exposed forests by
unloading, and cold, forests with calm winds by sublimation. Across all
sites, climate and canopy density exerted stronger controls on seasonal
snowfall partitioning than species-level differences between fir-,
spruce-, and pine-dominated forests.

Although the new model simulated canopy and subcanopy SWE well,
uncertainties remain in partitioning of snowfall between sublimation
losses and liquid water inputs to the forest floor, as direct flux
measurements have only been validated at one site in a previous study.
Measurements of snow interception, unloading, drip, and sublimation are
rarely made at hydrometeorological stations but would provide important
data to evaluate and refine process-level representations across
differing environments. Further research is also needed to determine
differences in the unloading relationships across a broad range in
climate and forest types. For instance, unloading can be strongly
affected by branch elasticity variations with temperature, tree age,
and/or tree species. Differing cohesion and adhesion in humid climates
may influence how snow is retained or shed from canopies compared to the
coefficients implemented here which were developed in a continental
climate. Moreover, all four study sites consisted primarily of mature
needleleaf forest canopies, and the transferability of these results to
more juvenile stands or forests with greater species diversity remains
an open area of research. Improved process-level measurements, combined
with continued model development, will help to identify these
uncertainties and support the transferability of canopy snow models
across the wide range of conditions found in snowy forests across the
globe.

\section{Acknowledgements}\label{acknowledgements}

We acknowledge financial support from the University of Saskatchewan
Dean's Scholarship, the Natural Sciences and Engineering Research
Council of Canada's Discovery Grants, the Canada First Research
Excellence Fund's Global Water Futures Programme, Environment and
Climate Change Canada, Alberta Innovates Water Innovation Program, the
Canada Foundation for Innovation's Global Water Futures Observatories
facility, and the Canada Research Chairs Programme. We thank Hannah
Koslowsky, Kieran Lehan, Lindsey Langs, James MacDonald, Michael
Solohub, Jacob Staines, Greg Galloway, Sean Carey, Rosy Tutton, David
Barrett, and Tyler De Jong for their help with data collection and Tom
Brown and Logan Fang for support of the CRHM platform.

\section{Data and Software Availability
Statement}\label{data-and-software-availability-statement}

The Cold Regions Hydrological Model Platform (CRHM) source code used in
this study is preserved at https://doi.org/10.5281/zenodo.17981431
(Pomeroy et al., 2025), available under the GPL-3.0 licence, and
developed openly at https://github.com/srlabUsask/crhmcode. Model
forcing data, model outputs, validation data, processed data, and
scripts to run the processing are available at
https://doi.org/10.5281/zenodo.17409551 with GPL-3.0 licence access
conditions.

\pagebreak

\section*{References}\label{references}
\addcontentsline{toc}{section}{References}

\phantomsection\label{refs}
\begin{CSLReferences}{1}{0}
\bibitem[\citeproctext]{ref-Andreadis2009}
Andreadis, K. M., Storck, P., \& Lettenmaier, D. P. (2009). Modeling
snow accumulation and ablation processes in forested environments.
\emph{Water Resources Research}, \emph{45}(5), 1--33.
\url{https://doi.org/10.1029/2008WR007042}

\bibitem[\citeproctext]{ref-Annandale2002}
Annandale, J., Jovanovic, N., Benadé, N., \& Allen, R. (2002). Software
for missing data error analysis of {Penman-Monteith} reference
evapotranspiration. \emph{Irrigation Science}, \emph{21}(2), 57--67.
\url{https://doi.org/10.1007/s002710100047}

\bibitem[\citeproctext]{ref-Barnhart2016}
Barnhart, T. B., Molotch, N. P., Livneh, B., Harpold, A. A., Knowles, J.
F., \& Schneider, D. (2016). Snowmelt rate dictates streamflow.
\emph{Geophysical Research Letters}, \emph{43}(15), 8006--8016.
\url{https://doi.org/10.1002/2016gl069690}

\bibitem[\citeproctext]{ref-Brutsaert1982}
Brutsaert, W. (1982). \emph{Evaporation into the {Atmosphere}: {Theory},
{History}, and {Applications}}. Dordrecht, Holland: Reidel.

\bibitem[\citeproctext]{ref-Calder1990}
Calder, I. R. (1990). \emph{Evaporation in the {Uplands}} (p. 148).
Wiley.

\bibitem[\citeproctext]{ref-NALCMS2020}
Canada Centre for Remote Sensing, Canada Centre for Mapping and Earth
Observation \& Natural Resources Canada. (2020). \emph{Land {Cover} of
{North America} at 30 meters} {[}Raster Digital Data{]}.

\bibitem[\citeproctext]{ref-Cebulski2025}
Cebulski, A. C., \& Pomeroy, J. W. (2025a). Theoretical underpinnings of
snow interception and canopy snow ablation parameterisations.
\emph{WIREs Water}, \emph{12}, e70010.
\url{https://doi.org/10.1002/wat2.70010}

\bibitem[\citeproctext]{ref-Cebulski2025a}
Cebulski, A. C., \& Pomeroy, J. W. (2025b). Snow interception
relationships with meteorology and canopy density. \emph{Hydrological
Processes}, \emph{39}(4), e70135.
\url{https://doi.org/10.1002/hyp.70135}

\bibitem[\citeproctext]{ref-Cebulski2025b}
Cebulski, A. C., \& Pomeroy, J. W. (2025c). Processes governing the
ablation of intercepted snow. \emph{Water Resources Research}, \emph{in
review}.

\bibitem[\citeproctext]{ref-Clark2021}
Clark, M. P., Vogel, R. M., Lamontagne, J. R., Mizukami, N., Knoben, W.
J. M., Tang, G., Gharari, S., Freer, J. E., Whitfield, P. H., Shook, K.
R., \& Papalexiou, S. M. (2021). The abuse of popular performance
metrics in hydrologic modeling. \emph{Water Resources Research},
\emph{57}(9), 1--16. \url{https://doi.org/10.1029/2020WR029001}

\bibitem[\citeproctext]{ref-Deschamps-Berger2025}
Deschamps-Berger, C., López-Moreno, J. I., Gascoin, S., Mazzotti, G., \&
Boone, A. (2025). Where snow and forest meet: A global atlas.
\emph{Geophysical Research Letters}, \emph{52}(10), e2024GL113684.
\url{https://doi.org/10.1029/2024GL113684}

\bibitem[\citeproctext]{ref-Ellis2010}
Ellis, C. R., Pomeroy, J. W., Brown, T., \& MacDonald, J. (2010).
Simulation of snow accumulation and melt in needleleaf forest
environments. \emph{Hydrology and Earth System Sciences}, \emph{14}(6),
925--940. \url{https://doi.org/10.5194/hess-14-925-2010}

\bibitem[\citeproctext]{ref-Essery2003}
Essery, R., Pomeroy, J. W., Parviainen, J., \& Storck, P. (2003).
Sublimation of snow from coniferous forests in a climate model.
\emph{Journal of Climate}, \emph{16}(11), 1855--1864.
\url{https://doi.org/10.1175/1520-0442(2003)016\%3C1855:SOSFCF\%3E2.0.CO;2}

\bibitem[\citeproctext]{ref-Fang2019}
Fang, X., Pomeroy, J. W., Debeer, C. M., Harder, P., \& Siemens, E.
(2019). Hydrometeorological data from marmot creek research basin,
canadian rockies. \emph{Earth System Science Data}, \emph{11}(2),
455--471. \url{https://doi.org/10.5194/essd-11-455-2019}

\bibitem[\citeproctext]{ref-Floyd2012}
Floyd, W. C. (2012). \emph{Snowmelt {Energy Flux Recovery During
Rain-on-Snow} in {Regenerating Forests}} (p. 180) {[}PhD thesis,
University of British Columbia{]}.
\url{https://doi.org/10.14288/1.0073024}

\bibitem[\citeproctext]{ref-Garnier1970}
Garnier, B. J., \& Ohmura, A. (1970). The evaluation of surface
variations in solar radiation income. \emph{Solar Energy}, \emph{13}(1),
21--34. \url{https://doi.org/10.1016/0038-092X(70)90004-6}

\bibitem[\citeproctext]{ref-Gelfan2004}
Gelfan, A. N., Pomeroy, J. W., \& Kuchment, L. S. (2004). Modeling
forest cover influences on snow accumulation, sublimation, and melt.
\emph{Journal of Hydrometeorology}, \emph{5}(5), 785--803.
\url{https://doi.org/10.1175/1525-7541(2004)005\%3C0785:MFCIOS\%3E2.0.CO;2}

\bibitem[\citeproctext]{ref-Groff2025}
Groff, T., \& Pomeroy, J. W. (2025). Snowmelt {Infiltration} and {Runoff
From Seasonally Frozen Hillslopes} in a {High Mountain Basin}.
\emph{Hydrological Processes}, \emph{39}(1), e70048.
\url{https://doi.org/10.1002/hyp.70048}

\bibitem[\citeproctext]{ref-Gupta2009a}
Gupta, H. V., Kling, H., Yilmaz, K. K., \& Martinez, G. F. (2009).
Decomposition of the mean squared error and {NSE} performance criteria:
{Implications} for improving hydrological modelling. \emph{Journal of
Hydrology}, \emph{377}(1-2), 80--91.
\url{https://doi.org/10.1016/j.jhydrol.2009.08.003}

\bibitem[\citeproctext]{ref-Harder2013}
Harder, P., \& Pomeroy, J. W. (2013). Estimating precipitation phase
using a psychrometric energy balance method. \emph{Hydrological
Processes}, \emph{27}(13), 1901--1914.
\url{https://doi.org/10.1002/hyp.9799}

\bibitem[\citeproctext]{ref-Hayashi2020}
Hayashi, M. (2020). Alpine hydrogeology: {The} critical role of
groundwater in sourcing the headwaters of the world. \emph{Groundwater},
\emph{58}(4), 498--510. \url{https://doi.org/10.1111/gwat.12965}

\bibitem[\citeproctext]{ref-Hedstrom1998}
Hedstrom, N. R., \& Pomeroy, J. W. (1998). Measurements and modelling of
snow interception in the boreal forest. \emph{Hydrological Processes},
\emph{12}(10-11), 1611--1625.

\bibitem[\citeproctext]{ref-Huerta2019}
Huerta, M. L., Molotch, N. P., \& McPhee, J. (2019). Snowfall
interception in a deciduous {Nothofagus} forest and implications for
spatial snowpack distribution. \emph{Hydrological Processes},
\emph{33}(13), 1818--1834.

\bibitem[\citeproctext]{ref-Immerzeel2020}
Immerzeel, W. W., Lutz, A. F., Andrade, M., Bahl, A., Biemans, H.,
Bolch, T., Hyde, S., Brumby, S., Davies, B. J., Elmore, A. C., Emmer,
A., Feng, M., Fernández, A., Haritashya, U., Kargel, J. S., Koppes, M.,
Kraaijenbrink, P. D. A., Kulkarni, A. V., Mayewski, P. A., \ldots{}
Baillie, J. E. M. (2020). Importance and vulnerability of the world's
water towers. \emph{Nature}, \emph{577}(7790), 364--369.
\url{https://doi.org/10.1038/s41586-019-1822-y}

\bibitem[\citeproctext]{ref-Isyumov1971}
Isyumov, N. (1971). \emph{An {Approach} to the {Prediction} of {Snow
Loads}} {[}PhD thesis{]}. The University of Western Ontario (Canada).

\bibitem[\citeproctext]{ref-Kim2017}
Kim, E., Gatebe, C., Hall, D., Newlin, J., Misakonis, A., Elder, K.,
Marshall, H. P., Hiemstra, C., Brucker, L., De Marco, E., Crawford, C.,
Kang, D. H., \& Entin, J. (2017). {NASA}'s snowex campaign: {Observing}
seasonal snow in a forested environment. \emph{2017 {IEEE} International
Geoscience and Remote Sensing Symposium ({IGARSS})}, 1388--1390.
\url{https://doi.org/10.1109/IGARSS.2017.8127222}

\bibitem[\citeproctext]{ref-Krinner2018}
Krinner, G., Derksen, C., Essery, R., Flanner, M., Hagemann, S., Clark,
M. P., Hall, A., Rott, H., Brutel-Vuilmet, C., Kim, H., Ménard, C. B.,
Mudryk, L., Thackeray, C., Wang, L., Arduini, G., Balsamo, G., Bartlett,
P., Boike, J., Boone, A., \ldots{} Zhu, D. (2018). {ESM-SnowMIP}:
{Assessing} snow models and quantifying snow-related climate feedbacks.
\emph{Geoscientific Model Development}, \emph{11}(12), 5027--5049.
\url{https://doi.org/10.5194/gmd-11-5027-2018}

\bibitem[\citeproctext]{ref-Langs2020}
Langs, L. E., Petrone, R. M., \& Pomeroy, J. W. (2020). A
{\(\delta\)18O} and {\(\delta\)2H} stable water isotope analysis of
subalpine forest water sources under seasonal and hydrological stress in
the {Canadian Rocky Mountains}. \emph{Hydrological Processes},
\emph{34}(26), 5642--5658. \url{https://doi.org/10.1002/hyp.13986}

\bibitem[\citeproctext]{ref-Leach2014}
Leach, J. A., \& Moore, R. D. (2014). Winter stream temperature in the
rain-on-snow zone of the {Pacific Northwest}: Influences of hillslope
runoff and transient snow cover. \emph{Hydrology and Earth System
Sciences}, \emph{18}(2), 819--838.
\url{https://doi.org/10.5194/hess-18-819-2014}

\bibitem[\citeproctext]{ref-Lopez-Moreno2014}
López-Moreno, J. I., Zabalza, J., Vicente-Serrano, S. M., Revuelto, J.,
Gilaberte, M., Azorin-Molina, C., Morán-Tejeda, E., García-Ruiz, J. M.,
\& Tague, C. (2014). Impact of climate and land use change on water
availability and reservoir management: {Scenarios} in the {Upper Aragón
River}, {Spanish Pyrenees}. \emph{Science of The Total Environment},
\emph{493}, 1222--1231.
\url{https://doi.org/10.1016/j.scitotenv.2013.09.031}

\bibitem[\citeproctext]{ref-Lumbrazo2022}
Lumbrazo, C., Bennett, A., Currier, W. R., Nijssen, B., \& Lundquist, J.
(2022). Evaluating multiple canopy-snow unloading parameterizations in
{SUMMA} with time-lapse photography characterized by citizen scientists.
\emph{Water Resources Research}, \emph{58}(6), 1--22.
\url{https://doi.org/10.1029/2021WR030852}

\bibitem[\citeproctext]{ref-Lundquist2021}
Lundquist, J. D., Dickerson-Lange, S., Gutmann, E., Jonas, T., Lumbrazo,
C., \& Reynolds, D. (2021). Snow interception modelling: {Isolated}
observations have led to many land surface models lacking appropriate
temperature sensitivities. \emph{Hydrological Processes}, \emph{35}(7),
1--20. \url{https://doi.org/10.1002/hyp.14274}

\bibitem[\citeproctext]{ref-MacDonald2010}
MacDonald, J. P. (2010). \emph{Unloading of {Intercepted Snow} in
{Conifer Forests}} (p. 93) {[}Msc Thesis{]}. University of Saskatchewan.

\bibitem[\citeproctext]{ref-Marks1992}
Marks, D., \& Dozier, J. (1992). Climate and energy exchange at the snow
surface in the {Alpine Region} of the {Sierra Nevada}: 2. {Snow} cover
energy balance. \emph{Water Resources Research}, \emph{28}(11),
3043--3054. \url{https://doi.org/10.1029/92WR01483}

\bibitem[\citeproctext]{ref-Marks1998}
Marks, D., Kimball, J., Tingey, D., \& Link, T. (1998). The sensitivity
of snowmelt processes to climate conditions and forest cover during
rain-on-snow: A case study of the 1996 {Pacific Northwest} flood.
\emph{Hydrological Processes}, \emph{12}(10-11), 1569--1587.
\url{https://doi.org/10.1002/(SICI)1099-1085(199808/09)12:10/11\%3C1569::AID-HYP682\%3E3.0.CO;2-L}

\bibitem[\citeproctext]{ref-Mazzotti2021}
Mazzotti, G., Webster, C., Essery, R., \& Jonas, T. (2021). Increasing
the physical representation of forest-snow processes in
coarse-resolution models: {Lessons} learned from upscaling
hyper-resolution simulations. \emph{Water Resources Research},
\emph{57}(5), 1--21. \url{https://doi.org/10.1029/2020WR029064}

\bibitem[\citeproctext]{ref-Nash1970}
Nash, J. E., \& Sutcliffe, J. V. (1970). River flow forecasting through
conceptual models part {I} --- {A} discussion of principles.
\emph{Journal of Hydrology}, \emph{10}(3), 282--290.
\url{https://doi.org/10.1016/0022-1694(70)90255-6}

\bibitem[\citeproctext]{ref-Pomeroy2022}
Pomeroy, J. W., Brown, T., Fang, X., Shook, K. R., Pradhananga, D.,
Armstrong, R., Harder, P., Marsh, C., Costa, D., Krogh, S. A.,
Aubry-Wake, C., Annand, H., Lawford, P., He, Z., Kompanizare, M., \&
Moreno, J. I. L. (2022). The cold regions hydrological modelling
platform for hydrological diagnosis and prediction based on process
understanding. \emph{Journal of Hydrology}, \emph{615}(128711), 1--25.
\url{https://doi.org/10.1016/j.jhydrol.2022.128711}

\bibitem[\citeproctext]{ref-CRHM2025}
Pomeroy, J. W., Brown, T., Fang, X., Shook, K. R., Pradhananga, D.,
Armstrong, R., Harder, P., Marsh, C., Costa, D., Krogh, S. A.,
Aubry-Wake, C., Annand, H., Lawford, P., He, Z., Kompanizare, M.,
Moreno, J. I. L., \& Cebulski, A. C. (2025). \emph{The cold regions
hydrological modelling platform {[}software{]}}. Zenodo.
\url{https://doi.org/10.5281/zenodo.17981431}

\bibitem[\citeproctext]{ref-Pomeroy1996}
Pomeroy, J. W., \& Dion, K. (1996). Winter radiation extinction and
reflection in a boreal pine canopy: {Measurements} and modelling.
\emph{Hydrological Processes}, \emph{10}(12), 1591--1608.
\url{https://doi.org/10.1002/(sici)1099-1085(199612)10:12\%3C1591::aid-hyp503\%3E3.0.co;2-8}

\bibitem[\citeproctext]{ref-Pomeroy2016a}
Pomeroy, J. W., Essery, R. L. H., \& Helgason, W. D. (2016). Aerodynamic
and radiative controls on the snow surface temperature. \emph{Journal of
Hydrometeorology}, \emph{17}(8), 2175--2189.
\url{https://doi.org/10.1175/JHM-D-15-0226.1}

\bibitem[\citeproctext]{ref-Pomeroy1995}
Pomeroy, J. W., \& Gray, D. M. (1995). \emph{Snowcover {Accumulation},
{Relocation} and {Management}} (NHRI Science Report No. 7, p. 144).
National Hydrology Research Institute, Environment Canada, Saskatoon,
Canada.

\bibitem[\citeproctext]{ref-Pomeroy2009}
Pomeroy, J. W., Marks, D., Link, T., Ellis, C. R., Hardy, J., Rowlands,
A., \& Granger, R. (2009). The impact of coniferous forest temperature
on incoming longwave radiation to melting snow. \emph{Hydrological
Processes}, \emph{23}, 2513--2525.
\url{https://doi.org/10.1002/hyp.7325}

\bibitem[\citeproctext]{ref-Pomeroy1998b}
Pomeroy, J. W., Parviainen, J., Hedstrom, N., \& Gray, D. M. (1998).
Coupled modelling of forest snow interception and sublimation.
\emph{Hydrological Processes}, \emph{12}(15), 2317--2337.
\url{https://doi.org/10.1002/(SICI)1099-1085(199812)12:15\%3C2317::AID-HYP799\%3E3.0.CO;2-X}

\bibitem[\citeproctext]{ref-Pomeroy2008}
Pomeroy, J. W., Rowlands, A., Hardy, J., Link, T., Marks, D., Essery,
R., Sicart, J. E., \& Ellis, C. R. (2008). Spatial variability of
shortwave irradiance for snowmelt in forests. \emph{Journal of
Hydrometeorology}, \emph{9}(6), 1482--1490.

\bibitem[\citeproctext]{ref-Pomeroy1993a}
Pomeroy, J. W., \& Schmidt, R. A. (1993). The use of fractal geometry in
modelling intercepted snow accumulation and sublimation. \emph{Eastern
Snow Conference}, \emph{50}, 231--239.

\bibitem[\citeproctext]{ref-Pypker2005}
Pypker, T. G., Bond, B. J., Link, T. E., Marks, D., \& Unsworth, M. H.
(2005). The importance of canopy structure in controlling the
interception loss of rainfall : {Examples} from a young and an
old-growth {Douglas-fir} forest. \emph{Agricultural and Forest
Meteorology}, \emph{130}, 113--129.
\url{https://doi.org/10.1016/j.agrformet.2005.03.003}

\bibitem[\citeproctext]{ref-Rasouli2019}
Rasouli, K., Pomeroy, J. W., Janowicz, J. R., Williams, T. J., \& Carey,
S. K. (2019). A long-term hydrometeorological dataset (1993--2014) of a
northern mountain basin: {Wolf} creek research basin, yukon territory,
canada. \emph{Earth System Science Data}, \emph{11}(1), 89--100.
\url{https://doi.org/10.5194/essd-11-89-2019}

\bibitem[\citeproctext]{ref-Roesch2001}
Roesch, A., Wild, M., Gilgen, H., \& Ohmura, A. (2001). A new snow cover
fraction parameterization for the {ECHAM4 GCM}. \emph{Climate Dynamics},
\emph{17}(12), 933--946. \url{https://doi.org/10.1007/s003820100153}

\bibitem[\citeproctext]{ref-Rojas-Heredia2024}
Rojas-Heredia, F., Revuelto, J., Deschamps-Berger, C., Alonso-González,
E., Domínguez-Aguilar, P., García, J., Pérez-Cabello, F., \&
López-Moreno, J. I. (2024). Snow depth distribution in canopy gaps in
central pyrenees. \emph{Hydrological Processes}, \emph{38}(11), e15322.
\url{https://doi.org/10.1002/hyp.15322}

\bibitem[\citeproctext]{ref-Rutter2009}
Rutter, N., Essery, R., Pomeroy, J. W., Altimir, N., Andreadis, K. M.,
Baker, I., Barr, A., Bartlett, P., Boone, A., Deng, H., Douville, H.,
Dutra, E., Elder, K., Ellis, C. R., Feng, X., Gelfan, A., Goodbody, A.,
Gusev, Y., Gustafsson, D., \ldots{} Yamazaki, T. (2009). Evaluation of
forest snow processes models ({SnowMIP2}). \emph{Journal of Geophysical
Research: Atmospheres}, \emph{114}(D6), 10--18.
\url{https://doi.org/10.1029/2008JD011063}

\bibitem[\citeproctext]{ref-Sanmiguel-Vallelado2017}
Sanmiguel-Vallelado, A., Morán-Tejeda, E., Alonso-González, E., \&
López-Moreno, J. I. (2017). Effect of snow on mountain river regimes: An
example from the {Pyrenees}. \emph{Frontiers of Earth Science},
\emph{11}(3), 515--530. \url{https://doi.org/10.1007/s11707-016-0630-z}

\bibitem[\citeproctext]{ref-Satterlund1970}
Satterlund, D. R., \& Haupt, H. F. (1970). The disposition of snow
caught by conifer crowns. \emph{Water Resources Research}, \emph{6}(2),
649--652. \url{https://doi.org/10.1029/WR006i002p00649}

\bibitem[\citeproctext]{ref-Schmidt1990}
Schmidt, R. A., \& Pomeroy, J. W. (1990). Bending of a conifer branch at
subfreezing temperatures: Implications for snow interception.
\emph{Canadian Journal of Forest Research}, \emph{20}(8), 1251--1253.
\url{https://doi.org/10.1139/x90-165}

\bibitem[\citeproctext]{ref-Shook2011}
Shook, K., \& Pomeroy, J. (2011). Synthesis of incoming shortwave
radiation for hydrological simulation. \emph{Hydrology Research},
\emph{42}(6), 433--446. \url{https://doi.org/10.2166/nh.2011.074}

\bibitem[\citeproctext]{ref-Staines2022}
Staines, J. (2021). \emph{Spatial {Relationships Between Trees} and
{Snow} in a {Cold Regions Montane Forest}} (Msc Thesis September; pp.
1--121). University of Saskatchewan.

\bibitem[\citeproctext]{ref-Staines2023}
Staines, J., \& Pomeroy, J. W. (2023). Influence of forest canopy
structure and wind flow on patterns of sub-canopy snow accumulation in
montane needleleaf forests. \emph{Hydrological Processes},
\emph{37}(10), 1--19. \url{https://doi.org/10.1002/hyp.15005}

\bibitem[\citeproctext]{ref-Storck2002}
Storck, P., Lettenmaier, D. P., \& Bolton, S. M. (2002). Measurement of
snow interception and canopy effects on snow accumulation and melt in a
mountainous maritime climate, {Oregon}, {United States}. \emph{Water
Resources Research}, \emph{38}(11), 1--16.
\url{https://doi.org/10.1029/2002wr001281}

\bibitem[\citeproctext]{ref-Stull2017}
Stull, R. B. (2017). \emph{Practical meteorology: {An} algebra-based
survey of atmospheric science.} (1.02b ed., pp. 1--944). University of
British Columbia.

\bibitem[\citeproctext]{ref-Thackeray2014}
Thackeray, C. W., Fletcher, C. G., \& Derksen, C. (2014). The influence
of canopy snow parameterizations on snow albedo feedback in boreal
forest regions. \emph{Journal of Geophysical Research: Atmospheres},
\emph{119}(16), 9810--9821. \url{https://doi.org/10.1002/2014JD021858}

\bibitem[\citeproctext]{ref-Viviroli2020}
Viviroli, D., Kummu, M., Meybeck, M., Kallio, M., \& Wada, Y. (2020).
Increasing dependence of lowland populations on mountain water
resources. \emph{Nature Sustainability}, \emph{3}(11), 917--928.
\url{https://doi.org/10.1038/s41893-020-0559-9}

\bibitem[\citeproctext]{ref-Wang2016}
Wang, L., Cole, J. N. S., Bartlett, P., Verseghy, D., Derksen, C.,
Brown, R., \& von Salzen, K. (2016). Investigating the spread in surface
albedo for snow-covered forests in {CMIP5} models. \emph{Journal of
Geophysical Research: Atmospheres}, \emph{121}(3), 1104--1119.
\url{https://doi.org/10.1002/2015JD023824}

\bibitem[\citeproctext]{ref-Watanabe1964}
Watanabe, S., \& Ozeki, J. (1964). Study of fallen snow on forest trees
({II}). {Experiment} on the snow crown of the {Japanese} cedar.
\emph{Japanese Government Forest Experiment Station Bulletin},
\emph{169}, 121--140.

\bibitem[\citeproctext]{ref-Xiao2016}
Xiao, Q., \& McPherson, E. G. (2016). Surface water storage capacity of
twenty tree species in {Davis}, {California}. \emph{Journal of
Environmental Quality. 45: 188-198}, \emph{45}, 188--198.
\url{https://doi.org/10.2134/jeq2015.02.0092}

\bibitem[\citeproctext]{ref-Zhang2018}
Zhang, Y., Sherstiukov, A. B., Qian, B., Kokelj, S. V., \& Lantz, T. C.
(2018). Impacts of snow on soil temperature observed across the
circumpolar north. \emph{Environmental Research Letters}, \emph{13}(4),
044012. \url{https://doi.org/10.1088/1748-9326/aab1e7}

\end{CSLReferences}




\end{document}
